\capitulo{6}{Trabajos relacionados}
Los sistemas de recomendación son una herramienta muy utilizada actualmente, marcando un punto de inflexión tanto  en el comercio electrónico   como  en las diferentes webs y apps, tales como Youtube, Spotify... 
Sin embargo, en relación a diferentes proyectos de Trabajo Final de Grado, éste es el primer proyecto orientado a dicho campo. 

Entre las diferentes plataformas, podemos especificar:

\section{Amazon}
Amazon utiliza el sistema de recomendación "ítem to ítem colaborative filtering" inventado por la propia empresa en 1998 ante el problema de los filtros colaborativos basados en usuarios:
\begin{itemize}
\item Fallo en la recomendación al disponer de una gran cantidad de elementos y pocas votaciones
\item Gran coste en el cálculo de las similitudes de usuarios. 
\item Cambio de perfil de usuarios de forma rápida.\nocite{wiki:filter}
\end{itemize}
Por otro lado, utiliza un sofware dirigido por un equipo de trabajadores encargado de enviar ofertas específicas a los diferentes usuarios según las últimas compras realizadas.\nocite{manu:amazon}
\section{Spotify}
Spotify  utiliza 3 sistemas de recomendación diferentes para mejorar las recomendaciones de sus canciones en Discovery Weekly: 
\begin{itemize}
\item Filtro colaborativo basado en usuarios. 
\item Modelo basado en el audio en bruto: Sistema de recomendación que utiliza una red neuronal (comvolutional neural network) para obtener las características musicales de las canciones escogidas por el usuario. 
\item NLP Sistema de procesamiento de lenguaje natural para encontrar nuevas tendencias musicales con las que poder trabajar. \cite{isaac:spotify}
\end{itemize}



\section{Youtube}
El sistema de recomendación de Youtube es bastante simple, ya que utiliza un filtro colaborativo basado en memoria (basado en usuarios). Sin embargo, lo filtra mediante calificaciones de forma que se mejora la ponderación del vídeo en los siguientes aspectos: 
\begin{itemize}
\item Regionalización: Filtrando el FC en el área geográfica en donde se encuentre el usuario o  en  el lenguaje de las búsquedas previas realizadas. \nocite{md:youtube}
\item Temporización: El sistema de recomendación de Youtube favorece los vídeos con una visualización de mayor duración, sin importar el porcentaje del vídeo reproducido. \nocite{anali:youtube}
\item Número de vídeos vistos por los usuarios a partir de uno determinado en una misma sesión. \nocite{hoots:youtube}
\item Determina si un vídeo gusta o no a los clientes utilizando inteligencia artificial para observar los comentarios inscritos en el vídeo. 
\end{itemize}
