\capitulo{3}{Conceptos teóricos}
\section{Sistemas de Recomendación}
Un sistema de recomendación es aquella herramienta encargada de recopilar, filtrar y devolver un conjunto de datos destinados para ofrecer al consumidor una orientación en base a sus gustos o en base a los gustos de otros usuarios cuyos aspectos introducidos coincidan.\nocite{wiki:recomendacion} \\
Podemos clasificar los sistemas de recomendación en dos grandes grupos: los filtros colaborativos y los sistemas basados en contenidos.
\subsection{Filtro colaborativo}
El filtro colaborativo es una clase particular de los sistemas de recomendación que requiere únicamente de la matriz de utilidad que expresa las valoraciones o ratings que los usuarios otorgan a los productos. El sistema funciona de la siguiente forma: \nocite{andres_gonzalez_sistemas_2014}
\begin{enumerate}
\item El usuario rellena el formulario con los ratings de aquellos elementos conocidos.
\item El sistema recopila los datos que son utilizados para construir las recomendaciones. \nocite{upf:recomendacion}
\item El sistema propone a un usuario aquellos items potencialemnete de su interés que todavía no ha consumido. \cite{wiki:filtrado}
\end{enumerate}
Una explicación más sencilla sería en que el sistema ofrece a un usuario lo más popular entre aquellos con sus mismos gustos.\nocite{ucm:recomendacion} 
Existen dos clases de Sistemas de Recomendación con filtro colabroativo: 
\subsubsection{Basado en modelos}
La estimación de los items se basa en el rating de los items de los diferentes usuarios. 
Por ello, para su utilización, se crea inicialmente una tabla, en la que las filas son los diferentes usuarios, mientras que las columnas hacen referencia a los diferentes elementos.
Dicha estimación utiliza los algoritmos de aprendizaje automático. Entre los ejemplos de los algoritmos existentes, podemos indicar los modelos de regresión, redes neuronales y algoritmos genéticos.


\begin{table}[]
\centering
\caption{Tabla de ejemplo de filtro colaborativo basado en modelo}
\label{tab:1}
\rowcolors {2}{gray!35}{}
\begin{tabular}{ lrrrr }
\toprule
Example                & $item_{1}$ & $item_{2}$ & ... & $item_{n}$ \\ \midrule
$user_{1}$ & 5         & 2           & ...           & 4       \\ 
$user_{2}$ & 5         & 1            & ...           & 3       \\ 
... & 2        & 5           & ...           & 5       \\ 
$user_{n}$ & 4         & ?           & ...          & 3       \\ \bottomrule
\end{tabular}
\end{table}

Posteriormente se ajusta el modelo para proponer al usuario un valor recomendado para el item desconocido (?).
\\En este trabajo se ajustará un modelo de regresión lineal. Conocida la matriz de ratings que los usuarios han dado a los diferentes items, se estiman los parámetros de un modelo lineal que calcula el rating que el usuario j da al item i:\begin{equation}
\widehat{y_{ij}} = x^{i}\Theta ^{jT}
\end{equation}\\Donde las {x\_{i}}i=1,2,... representan el espacio de características de los items, y las {j}j=1,2,... representan el espacio de preferencias de los usuarios. La estimación de los parámetros se obtiene minimizando la siguiente función objetivo o función de coste:\\
\begin{equation}
min_{\Theta ^{(1)}...\Theta ^{(u)} x^{1}...x^{m}} \frac{1}{2}\sum_{(i,j):r(i,j)=1}^{ }(x^{(i)}(\Theta ^{(j)})^{T}-y^{i,j})^{2}+\frac{\lambda }{2}\sum_{i=1}^{m}\sum_{k=1}^{n}(x_{k}^{(i)})^{2}\sum_{j=1}^{u}\sum_{k=1}^{n}(\theta _{k}^{(j)})^{2}
\end{equation}
\\El parámetro  es un parámetro de regularización que permite modular el posible sobreajuste del modelo.  Este filtro colaborativo es un problema idéntico al problema matemático de aproximación “low-rank”.
\\  

\subsubsection{Basado en memoria}
La estimación de los items se basa en los rating del cliente con respecto a los items de los diferentes usuarios. El sistema busca las similitudes entre las preferencias de los usuarios para ofrecer una recomendación. 
A su vez, el sistema de recomendación con filtro colaborativo basado en memoria se puede subdividir en: 
\begin{enumerate}

\item  \textbf{Basado en Usuarios}\\ La estimación del  $item_{i}$ de un $usuario_{u}$ concreto se basa en los rating del $item_{i}$  de los usuarios  con mayor proximidad al $usuario_{u}$. \\Se han utilizado dos fórmulas diferentes para el cálculo del sistema de recomendación basado en usuarios: \\ \begin{itemize}

\item Similitud de coseno:\\\begin{equation}sim_{u,v}=\frac{\sum_{i\epsilon I}r_{u,i}r_{v,i}}{\sqrt{\sum_{i\epsilon I}r_{u,i}^{2}}\sqrt{\sum_{i\epsilon I}r_{v,i}^{2}}}
\end{equation}
\item Coeficiente de Pearson:\\ \begin{equation}sim_{u,v}=\frac{\sum_{i \in I}(r_{u,i}-\bar{r}_{u}) (r_{v,i}-\bar{r}_{v})}{\sqrt{\sum_{i \in I}(r_{u,i}-\bar{r}_{u})^{2}} \sqrt{\sum_{i \in I}(r_{v,i} - \bar{r}_{v})^{2}}}
\end{equation}
 \\La fórmula para estimar un ranking de un usuario es la siguiente : \begin{equation}
\hat{r}_{u,i}= \bar{r_{u}}+\frac{\sum_{v\in \bigcup }(r_{v,i}-\bar{r_{v}})sim_{u,v}}{\sum_{v\in \bigcup } \left | sim_{u,v} \right |} 
\end{equation}
\end{itemize}
\item \textbf{Basado en Productos}\\La estimación del  $item_{i}$ de un $usuario_{u}$ concreto se basa en los rating del $usuario_{u}$ a los items similares a $item_{i}$.\\ \begin{itemize}
\item Similitud de coseno : \\ \begin{equation}
sim_{i,j }=\frac{\sum_{u\in U}r_{u,i}r_{u,j}}{\sqrt{\sum_{u\in U}r^{2}_{u,i}}\sqrt{\sum_{u\in U}r^{2}_{u,j}}}
\end{equation}
\item Coeficiente de Pearson: \\ \begin{equation}
sim_{i,j}=\frac{\sum_{u \in U}(r_{u,i}-\bar{r}_{i}) (r_{u,j}-\bar{r}_{j})}{\sqrt{\sum_{u \in U}(r_{u,i}-\bar{r}_{i})^{2}} \sqrt{\sum_{u \in U}(r_{u,j} - \bar{r}_{j})^{2}}}
\end{equation} \\La fórmula para estimar un ranking en un sistema basado en productos es la siguiente:\nocite{ubu:recomendacion} \\ \begin{equation}
\hat{r}_{u,i}= \frac{\sum_{j\in I }r_{u,j}sim_{i,j}}{\sum_{j\in I } \left | sim_{i,j} \right |} 
\end{equation}
\end{itemize}
\end{enumerate}


\subsection{Filtrado basado en contenido}
Los filtros basados en contenidos requieren de más información que la matriz de utilidad o ratings mencionada anteriormente. En general este tipo de técnicas analizan las características de los usuarios y el contenido de los productos para encontrar patrones que permitan hacer recomendaciones de nuevos productos
 
