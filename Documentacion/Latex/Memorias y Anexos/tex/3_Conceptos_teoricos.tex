\capitulo{3}{Conceptos teóricos}
\section{Sistemas de Recomendación}
Un sistema de recomendación es aquella herramienta encargada de recopilar, filtrar y devolver un conjunto de datos destinados para ofrecer al consumidor una orientación en base a sus gustos o en base a los gustos de otros usuarios cuyos aspectos introducidos coincidan.\nocite{wiki:recomendacion} \\
Los sistemas de recomendación se subdividen en dos grupos: 
\subsection{Filtro colaborativo}
Técnica utilizada en los sistemas de recomendación basada en predicción de las preferencias basado en las selecciones de los usuarios. Para ello, el sistema funciona de la siguiente forma: \nocite{andres_gonzalez_sistemas_2014}
\begin{enumerate}
\item El usuario rellena el formulario con respecto a las preferencias de aquellos elementos conocidos.
\item El sistema recopila los datos y realiza las comparaciones con otros usuarios para determinar las similitudes. \nocite{upf:recomendacion}
\item El sistema informa de aquellos items en los que el cliente podría coincidir con el resto de usuarios pero no ha rellenado. \cite{wiki:filtrado}
\end{enumerate}
Una explicación más sencilla sería en que el sistema ofrece a un usuario lo más popular entre aquellos con sus mismos gustos.\nocite{ucm:recomendacion} 
Existen dos clases de Sistemas de Recomendación con filtro colabroativo: 
\subsubsection{Basado en modelos}
La estimación de los items se basa en el rating de los items de los diferentes usuarios. 
Por ello, para su utilización, se crea inicialmente una tabla, en la que las filas son los diferentes usuarios, mientras que las columnas hacen referencia a los diferentes elementos.

\begin{table}[]
\centering
\caption{Tabla de ejemplo de filtro colaborativo basado en modelo}
\label{tab:1}
\rowcolors {2}{gray!35}{}
\begin{tabular}{ lrrrr }
\toprule
Example                & $item_{1}$ & $item_{2}$ & ... & $item_{n}$ \\ \midrule
$user_{1}$ & 5         & 2           & ...           & 4       \\ 
$user_{2}$ & 5         & 1            & ...           & 3       \\ 
... & 2        & 5           & ...           & 5       \\ 
$user_{n}$ & 4         & ?           & ...          & 3       \\ \bottomrule
\end{tabular}
\end{table}

Posteriormente se ajusta el modelo para proponer al usuario un valor recomendado para el item desconocido (?).
\subsubsection{Basado en memoria}
La estimación de los items se basa en los rating del cliente con respecto a los items de los diferentes usuarios. El sistema, tras rellenar la tabla de usuarios/items busca las similitudes entre las preferencias de los usuarios para ofrecer una recomendación. 
A su vez, el sistema de recomendación con filtro colaborativo basado en memoria se puede subdividir en: 
\begin{enumerate}

\item  \textbf{Basado en Usuarios}\\ La estimación del  $item_{i}$ de un $usuario_{u}$ concreto se basa en los rating del $item_{i}$  de los usuarios  con mayor proximidad al $usuario_{u}$. \\Se han utilizado dos fórmulas diferentes para el cálculo del sistema de recomendación basado en usuarios: \\ \begin{itemize}

\item Similitud de coseno:\\\begin{equation}sim_{u,v}=\frac{\sum_{i\epsilon I}r_{u,i}r_{v,i}}{\sqrt{\sum_{i\epsilon I}r_{u,i}^{2}}\sqrt{\sum_{i\epsilon I}r_{v,i}^{2}}}
\end{equation}
\item Coeficiente de Pearson:\\ \begin{equation}sim_{u,v}=\frac{\sum_{i \in I}(r_{u,i}-\bar{r}_{u}) (r_{v,i}-\bar{r}_{v})}{\sqrt{\sum_{i \in I}(r_{u,i}-\bar{r}_{u})^{2}} \sqrt{\sum_{i \in I}(r_{v,i} - \bar{r}_{v})^{2}}}
\end{equation}
 \\La fórmula para estimar un ranking de un usuario es la siguiente : \begin{equation}
\hat{r}_{u,i}= \bar{r_{u}}+\frac{\sum_{v\in \bigcup }(r_{v,i}-\bar{r_{v}})sim_{u,v}}{\sum_{v\in \bigcup } \left | sim_{u,v} \right |} 
\end{equation}
\end{itemize}
\item \textbf{Basado en Productos}\\La estimación del  $item_{i}$ de un $usuario_{u}$ concreto se basa en los rating del $usuario_{u}$ a los items similares a $item_{i}$.\\ \begin{itemize}
\item Similitud de coseno : \\ \begin{equation}
sim_{i,j }=\frac{\sum_{u\in U}r_{u,i}r_{u,j}}{\sqrt{\sum_{u\in U}r^{2}_{u,i}}\sqrt{\sum_{u\in U}r^{2}_{u,j}}}
\end{equation}
\item Coeficiente de Pearson: \\ \begin{equation}
sim_{i,j}=\frac{\sum_{u \in U}(r_{u,i}-\bar{r}_{i}) (r_{u,j}-\bar{r}_{j})}{\sqrt{\sum_{u \in U}(r_{u,i}-\bar{r}_{i})^{2}} \sqrt{\sum_{u \in U}(r_{u,j} - \bar{r}_{j})^{2}}}
\end{equation} \\La fórmula para estimar un ranking en un sistema basado en productos es la siguiente:\nocite{ubu:recomendacion} \\ \begin{equation}
\hat{r}_{u,i}= \frac{\sum_{j\in I }r_{u,j}sim_{i,j}}{\sum_{j\in I } \left | sim_{i,j} \right |} 
\end{equation}
\end{itemize}
\end{enumerate}


\subsubsection{Basado en modelo}
La estimación de los items se basa en utilizar la minería de datos con algoritmos de aprendizaje automático. Entre los ejemplos de los algoritmos existentes, podemos indicar los clasificadores bayesianos, redes neuronales y algoritmos genéticos.\nocite{wiki:filtrado} 


\subsection{Filtrado basado en contenido}
Técnica utilizada en los sistemas de recomendación,  basada en asignar unos pesos a cada uno de los items conocidos por el usuario, de forma que se pueda utilizar el algoritmo para extraer y recomendar aquellos items que tengan mayor similitud con los gustos del usuario.
Su explicación se basaría en ofrecer al usuario aquello que se relaciona más con lo que le ha gustado. 

 
\subsection{Filtrado híbrido}

