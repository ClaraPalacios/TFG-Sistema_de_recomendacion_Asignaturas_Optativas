\newcommand{\grad}{$^{\circ}$}
\capitulo{1}{Introducción}
%\maketitle
\nonzeroparskip
Este proyecto consiste en resolver el problema de la elección de las asignaturas optativas del último curso entre los estudiantes del grado de Ingeniería Informática. \\

Dado el desconocimiento de la materia impartida en 4º curso y la falta de visión de la relación entre la misma, un gran porcentaje de estudiantes universitarios consideran que su elección en alguna de las asignaturas no ha sido la correcta. \\

Por ese motivo, para evitar confusiones por parte de los alumnos, se creará una aplicación mediante la cual, el usuario rellenará un  cuestionario anónimo, introduciendo  una calificación por cada una de las asignaturas cursadas, de forma que el sistema sea capaz de indicarle cuál de las optativas se ajusta más a su criterio. 

Sin embargo, a pesar de poner en práctica diferentes filtros colaborativos, los resultados pueden mejorarse, ya que el número de datos recopilados hasta el momento son mínimos con respecto a los sistemas de recomendación utilizados en grandes empresas, de forma que la recomendación ofrecida al usuario es simplemente orientativa. 