\capitulo{4}{Técnicas y herramientas}

\section{Metodología Ágil}

La metodología ágil es aquella forma de toma de decisiones en los proyectos software basado en un desarrollo iterativo, evaluando las necesidades y tareas necesarias a la par de la realización, para añadir funcionalidad en el proyecto.\cite{wiki:desarrollo}\\ 
\subsection{KANBAN}
Sistema de información basado en la metodología ágil. Utiliza tarjetas para representar la información de forma visual, mejorando la distribución de trabajo y la organización del mismo. \cite{ kanban:metodo}
La representación Kanban se desarrolla en un tablero, asignando tareas a los diferentes miembros, utilizando tarjetas para indicar las diferentes etapas en las que se encuentran las subtareas del proyecto. \cite{wiki:kanban}
 
\subsection{SCRUM}
Modelo de referencia basado en metodología ágil basado en un desarrollo incremental que permite modificar tareas e ideas una vez comenzado el proyecto. \nocite{agile:scrum}
Hemos seleccionado esta opción, ya que entre sus ventajas podemos indicar: 
\begin{itemize}
\item Flexibilidad ante las diferentes necesidades y posibles cambios. \nocite{proyectos:scrum}
\item Entrega progresiva del proyecto. 
\item Control en todas las etapas del proyecto. 
\end{itemize}
\section{Repositorio de código y control de versiones}
\subsection{VersionOne}
\nonzeroparskip
Plataforma compacta basada en gestión ágil, está estructurada para admitir diferentes metodolgías ágiles, tales como Scrum, Kanban, Lean o XP, permitiendo la rápida escalabilidad de los proyectos. \cite{versionOne:soporte} \\
Como ventajas podemos indicar que: 
\begin{itemize}
\item Fácil de utilizar. 
\item Gran cantidad de métricas ágiles 
\item Fácilmente escalable. 
\end{itemize}
Sin embargo, es un software de pago, ya que las utilidades necesarias para el Trabajo Final de Carrera no están incluidas en la versión gratuita por lo que ha sido descartada esta opción. 
\subsection{Bitbucket}
Plataforma de almacenamiento de código que utiliza el sistema de control de versiones \nocite{wiki:bitbucket} basado en gestión ágil.
\begin{itemize}
\item Ventajas
\begin{itemize}
\item Número ilimitado de creación de repositorios  privados gratuitos. 
\item Control de versiones Git y Mercurial
\item Software de comunicación entre los diferentes colaboradores de un proyecto.\nocite{redes:bitbucket} 
\end{itemize}
\end{itemize}
\begin{itemize}
\item Inconvenientes
\begin{itemize}
\item Número limitado de colaboradores en un proyecto gratuito. 
\item  Exclusión de un soporte para el sistema operativo Linux.
\item El coste por tener un mayor número de colaboradores en un proyecto es muy superior a Github-cuyo coste se basa en el número de repositorios privados existentes.\nocite{rocreguant:bitbucket} 
\end{itemize}
\end{itemize}


\subsection{Gitlab}
Sistema de control de versiones basado en gestión ágil \nocite{wiki:gitLab} y gestor de repositorio de código con licencia MIT-de software libre permisivo-\nocite{wiki:MIT}
\begin{itemize}
\item Ventajas
\begin{itemize}
\item Facilidad en adjuntar archivos en los issues.
\item Protección de ramificaciones frente a cambios utilizando el sistema de niveles de autorización. \cite{openbinars:gitlab}
\item Posibilidad de crear repositorios privados de forma gratuita. 
\item No hay limitaciones ni en el número de repositorios privados ni en el número de colaboradores. 
\end{itemize}
\end{itemize}
\begin{itemize}
\item Inconvenientes
\begin{itemize}
\item Menor cantidad de miembros en la comunidad. 
\item Gestor de repositorio totalmente nuevo, por lo que se debería estudiar su funcionamiento. \nocite{platzi:gitlab}
\end{itemize}
\end{itemize}

\subsection{SourceForge}
Sitio web  de sofware libre para almacenar y compartir proyectos de código abierto.
\begin{itemize}
\item Ventajas
\begin{itemize}
\item Gran número de proyectos alojados.
\end{itemize}
\end{itemize}
\begin{itemize}
\item Inconvenientes
\begin{itemize}
\item Monetización de la descarga de los proyectos basándose en la inclusión de anuncios publicitarios en los instaladores.\cite{wiki:SouceForge}
\item Necesidad de aprendizaje 
\end{itemize}
\end{itemize}
\subsection{Github}
GitHub es una plataforma de desarrollo colaborativo de código para alojar proyectos utilizando el sistema de control de versiones Git.\nocite{git:gith} Permite gestionar los cambios de código indicando qué se ha modificado, añadido o eliminado en cualquier commit realizado. \\Entre sus ventajas e inconvenientes, podemos indicar: 
\begin{itemize}
\item Ventajas
\begin{enumerate}
\item Servicio gratuito en los repositorios públicos. 
\item No es necesario realizar copias de seguridad del código, ya que utiliza un sistema de control de versiones. 
\item Compatible con diferentes sistemas operativos (Windows, Linux, OS...). \nocite{gits:gitHub}
\end{enumerate}
\end{itemize}
\begin{itemize}
\item Inconvenientes
\begin{enumerate}
\item Los repositorios privados no son gratuitos. 
\end{enumerate}
\end{itemize}
Sin embargo, a pesar del inconveniente del pago en caso de desear un repositorio privado, las ventajas de la utilización de GitHub tienen un mayor peso, además de haber sido la plataforma utilizada en los cursos académicos del grado de Ingeniería Informática. 
\section{Gestores de Tareas}
\subsection{Trello}
Herramienta para la gestión y seguimiento\nocite{xelso:trello} de las tareas en tiempo real. Para acceder a dicha herramienta, se debe crear una cuenta en Trello.com. Sin embargo, para evitar el uso masivo de herramientas y aplicaciones innecesarias en nuestro proyecto, descartamos esta opción. 

\subsection{Hojas de Cálculo}
Una hoja de cálculo es una herramienta  para la manipulación de datos y gráficos en hojas divididas en celdas. Se podría  Sin embargo, ante la falta de orden y de un correcto seguimiento, se ha descartado dicha herramienta desde un comienzo. 
\subsection{Zenhub}
ZenHub es una extensión integrada en Chrome orientada a la gestión y el control de proyectos utilizando pizarras o columnas en las que se ordenan las issues creadas para mejorar la clasificación de las mismas. 
Utilizaremos esta herramienta para la gestión y el seguimiento de las diferentes tareas, creando nuevas Pipelines  y utilizando las existentes. Entre ellas, podemos encontrar: 
\begin{itemize}
\item ICEBOX \\ Issues que se hayan comenzado, y sin embargo, por cualquier motivo, se deben dejar paradas. 
\item Backlog \\ Issues que puede que  se vayan a desarrollar más adelante.
\item To do \\Isues pendiendtes por desarrollar. 
\item In progress \\Issues que están en desarrollo en ese mismo momento
\item Clossed \\Issues que se han terminado y han sido cerradas. 
\end{itemize}
\section{Editores de Texto}
\subsection{Microsoft Word}
Microsoft Word es un programa de la empresa Microsoft creado para la edición de texto\cite{wiki:Word}
\begin{itemize}
\item Ventajas
\begin{itemize}
\item Interfaz gráfica que permite una mejor comprensión de la herramienta para los usuarios nóveles. \nocite{elblogdel1:Word}
\item Sencillez ante la edición de texto así como la utilización de los diferentes formatos.
\end{itemize}
\end{itemize}
\begin{itemize}
\item Inconvenientes
\begin{itemize}
\item Fallo de seguridad en los complementos la herramienta que permite el robo de archivos mediante la introducción de un documento con código oculto.\cite{monografias:Word}
\item Limitación de la capacidad para la inclusión y el tratamiento de las imágenes.
\item Problemática en el mantenimiento del mismo formato en todo el documento.
\end{itemize}
\end{itemize}
\subsection{OpenOffice}
Paquete de sofware de código abierto destinado para el procesamiento y la edición de texto. 
\begin{itemize}
\item Ventajas
\begin{itemize}
\item Al igual que Microsoft Word, consta de una interfaz gráfica  fácil de utilizar en la edición y tratamiento de texto, gráficos y tablas. 
\item No tiene coste de licencia-al contrario que Microsoft Word.\nocite{apache:office}
\item Multiplataforma. 
\end{itemize}
\end{itemize}
\begin{itemize}
\item Inconvenientes
\begin{itemize}
\item Lentitud en el procesamiento, edición y guardado de archivos\nocite{theclandia:office}
\item Incompatibilidad con Microsoft Word, por lo que varias de las acciones disponibles en OpenOffice Writer no se reconocen en Microsoft Word; acciones tales como es el pegado de una tabla.
\item Al igual que ocurre en Microsoft Word, puede ser dificultoso mantener el mismo formato en todo el documento.
\end{itemize}
\end{itemize}
\subsection{\LaTeX}
Sistema de edición y procesamiento de texto basado en comandos de TeX\footnote{Sistema de tipografía desarrollado en 1985 por Donald E. Knuth, de licencia libre, es utilizado en entornos académicos por el alto número de funcionalidades que ofrece\cite{wiki:Tex}.}\\
Los archivos de \LaTeX tienen la extensión .tex, cuyo contenido debe ser compilado de forma previa a su visualización. En este proyecto, la compilación será PDFLaTeX, para obtener el formato PDF del fichero escrito. 
\begin{itemize}
\item Ventajas
\begin{itemize}
\item Sencillez en la utilización del mismo formato durante el proyecto, referencias cruzadas y numeración (Documento estructurado)
\item Calidad en la edición de texto y funciones matemáticas. 
\item Multiplataforma.
\item Es una herramienta portable y gratuita.
\end{itemize}
\end{itemize}
\begin{itemize}
\item Inconvenientes
\begin{itemize}
\item Es una herramienta compleja, ya que requiere un periodo previo a su utilización de aprendizaje de comandos básicos. 
\item Dificultad en la introducción de imágenes  y bibliografías.
\item Necesidad de compilación por cada cambio realizado para observar los resultados finales. \nocite{aq:LaTex} 
\end{itemize}
\end{itemize}
\LaTeX  será el sistema de edición de texto escogido, ya que, a pesar de la dificultad de su aprendizaje, hemos considerado que tiene mayor peso las ventajas que los inconvenientes propuestos.

\section{Gestores de Referencias Bibliográficas}
\subsection{Mendeley}
Mendeley es un sofware  gratuito destinado a la gestión de referencias bibliográficas. Compatible con diferentes versiones de Microsoft Word, OpenOffice y BibTex tiene como característica principal la sincronización de las referencias tanto en el equipo  propio como en Web. \nocite{ucm:mendeley}
Entre sus ventajas e inconvenientes, podemos destacar: 
\begin{itemize}
\item Multiplataforma
\item Organización de las referencias. 
\item Compartir referencias bibliográficas entre diferentes usuarios. 

\end{itemize}
\subsection{Zotero}
Software libre de código abierto destinado a la gestión de referencias bibliográficas. \cite{bibl:zotero} Utilizado en Firefox, Chrome, Safari y Opera, sigue cinco principios básicos: 
\begin{enumerate}
\item Recopilar información. 
\item Organizar los recursos en la biblioteca. 
\item Citar las referencias bibliográficas de manera automática en la edición de texto. 
\item Sincronización de la biblioteca en un servidor. 
\item Colaboración al permitir compartir las bibliotecas creadas con el resto de usuarios. 
\end{enumerate}
\begin{itemize}
\item Ventajas
\begin{itemize}
\item Multiplataforma
\item Extracción automática de las citas en la edición de texto. 
\item Sincronización con diferentes editores de texto (Microsoft Word, OpenOffice y LibreOffice)\cite{wiki:Zotero}
\item Importación de metadatos de  un amplio número de soportes.
\item Reconoce bibliotecas en formato BibTex.
\end{itemize}
\end{itemize}
\begin{itemize}
\item Inconvenientes
\begin{itemize}
\item No dispone de lector de PDF. 
\item Necesidad de utilizar aplicaciones de terceros para obtener todas las funcionalidades disponibles. 
\end{itemize}
\end{itemize}
\subsection{BibTex}
\nonzeroparskip
La herramienta elegida para la bibliografía es BibTex, diseñada como una utilidad de apoyo bibliográfico para LateX. \\
\nonzeroparskip
Para utilizarlo, se emplean los ficheros.bib en los que se encuentre la bibliografía necesaria (librerías), de forma que BibTex añadirá a la bibliografía las librerías que hayan sido citadas en el documento. \\
\nonzeroparskip
Además, para obtener los datos de la bibliografia, basta con abrir una pestaña de  "school.google.es, marcar en Configuración la opción "Mostrar enlaces para importar citas a BibTex". De esta forma, una vez encontrada la página deseada, tan sólo es necesario copiar la bibliografía que se muestre y guardarla en nuestra librería.\cite{perez1968titulo} 

\section{Editores de código}
\section{Recogida de datos}
La recogida de datos se ha realizado mediante un cuestionario anónimo difundido entre los Alumnos que hayan cursado la carrera del Grado de Ingeniería Informática en Burgos. Entre las diferentes opciones que existían destacamos: 
\subsection{SurveyMonkey}
Herramienta de creación de cuestionarios online con versión gratuita para realizar: 
\begin{itemize}
\item 15 tipos de preguntas diferentes. 
\item 10 preguntas máximo por cuestionario. 
\item Un máximo de 100 respuestas.  
\end{itemize}
Hemos descartado dicha herramienta, ya que en la versión gratuita no permite la exportación de datos en ningún soporte, además de la limitación del número de preguntas por cuestionario. 
\subsection{Zoho Survey}
Herramienta de creación de cuestionarios online con una versión gratuita con las siguientes características: 
\begin{itemize}
\item Número ilimitado de cuestionarios. 
\item 15 preguntas posibles por cuestionario. 
\item 150 respuestas posibles por cuestionario. 
\end{itemize}
Sin embargo, hemos descartado dicha herramienta por la limitación del número de preguntas por cuestionario. 
\subsection{Eval \& Go}
Herramienta de creación de cuestionarios online con posibilidad de creación de una cuenta gratuita con las siguientes características: 
\begin{itemize}
\item Número de encuestas ilimitadas por cuenta. 
\item Posibilidad de exportación de los  datos  recogidos en formato Excel o CSV. 
\end{itemize}
En un principio consideramos utilizar dicha herramienta, sin embargo, tiene un máximo de 150 respuestas al mes, y al necesitar el mayor número de respuestas de alumnos, tendríamos esa limitación,por lo que no podríamos recoger el máximo número de respuestas. \nocite{carl:encuestas}
\subsection{TypeForm}
Es una herramienta de creación de cuestionarios online anónimos, con una versión gratuita con las siguientes características: 
\begin{itemize}
\item Número ilimitado de preguntas, respuestas y cuestionarios por cuenta. 
\item Posibilidad de utilizar alguna plantilla predefinida o desarrollar una propia. 
\item Formularios responsive (adaptados a los diferentes dispositivos con los que se interaccione) 
\item Exportación de los datos en formato Excel en Drive o Google+, de forma que se puedan ver los resultados en tiempo real. 
\end{itemize}
El cuestionario realizado se encuentra en la siguiente dirección \url{https://clarapalacios.typeform.com/to/RQRRfY}. Es un cuestionario sencillo,  en donde se da una explicación breve del funcionamiento del mismo, se solicitan las ponderaciones por cada asignatura (obligatorias para los 3 primeros cursos académicos y optativas para el 4º año). No se permite que mismo usuario responda más de una vez al cuestionario, ya que, a pesar de ser anónimo, se almacena un identificador para evitar que un mismo alumno pueda rellenar varias veces el cuestionario.  	


\section{Integración de funcionalidades del cuestionario}
Los datos se guardarán en un documento Excel-sincronizado en Drive-  llamado  \textbf{DatosTFG\_SistemasRecomendacion}, almacenado en la siguiente dirección: \url{https://docs.google.com/spreadsheets/d/1Dtu6HPKu_d0zToz2b16dhfz4k7OluqAASYbveHUfSls/edit?usp=sharing} \\En un principio, teníamos pensado utilizar la librería de Python  OpenPyxl  de forma que se nos permitirá leer y almacenar en una matriz los datos recogidos del Excel. Sin embargo, nos encontramos problemas con la lectura, por lo que hemos decidido utilizar el API de GoogleDrive, descargando un fichero .json para la obtener la clave privada y almacenarla en el ordenador, en el directorio en donde se encuentre el código en Python de la lectura de datos de cuestionario anónimo. \\Además, serán necesarias dos librerías, \textbf{oauth2client} y \textbf{gspread}, pudiéndose instalar con un único código en cmd (Símbolo de Sistema de Windows)  mediante el comando \textbf{``pip install gspread oauth2client''}. \nocite{twilio:api}con la versión de pip 8.1.2, a pesar de no ser la más actual existente. 

\subsection{Funcionamiento del Google Drive API }
La activación e integración del fichero situado en Drive para poder utilizarlo desde Jupyter se ha realizado de la siguiente manera: 
\begin{itemize}
\item Iniciar sesión en Google Drive
\item Activar en la consola de APIs de Google la API. 
\item Crear (o en caso de tener uno creado, abrir) un nuevo proyecto. Hay un máximo de 11 proyectos. 
\item Activar las credenciales de Google Drive. 
\item Crear credenciales para Servidor Web para acceder a la aplicación. 
\item Dar un nombre al proyecto y asignarle un rol (Project-Editor)
\item Descargar el fichero .JSON correspondiente. 
\item Copiar el fichero (renombrado a client\_secret.json) en el directorio en donde obtengamos los datos en python del Excel. 
\item Copiar el client\_email situado en el fichero, y compartirlo con el fichero en Drive.
\item Instalar ambas librerías, e importarlas en el Notebook  de Python. 
\end{itemize}

\section{Herramienta para el desarrollo}

