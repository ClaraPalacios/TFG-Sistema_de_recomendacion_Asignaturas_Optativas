\capitulo{8}{Conclusiones y Líneas de trabajo futuras}
En este capítulo indicaremos las conclusiones obtenidas del proyecto, de su realización y de las posibilidades, así como las líneas por las que se podría seguir su desarrollo. 
\section{Conclusiones}
Este proyecto ha sido novedoso, tanto por ser un  sistema de recomendación, como su implementación  y desarrollo. Ha sido un reto el tener el servidor en Cloud y poder acceder a los datos desde una aplicación de escritorio. En este proyecto también sería posible utilizar un servidor local, en caso de fallo de conexión a Internet. 
A pesar de la dificultad de adaptación del código para que almacene los datos recogidos a través de la interfaz y los almacene en la Base de Datos, se  ha conseguido dicha adaptación de la forma más limpia posible. 
Por otro lado, se ha deseado implementar una mayor variedad de sistemas de recomendación con diferentes algoritmos, sin embargo, a falta de tiempo, únicamente se han desarrollado dos. 
También se ha deseado tener el código de los sistemas de recomendación alojados en Cloud, pero, ante la falta de recursos de forma gratuita disponibles en el servidor, no hemos querido sobrecargarlo con peticiones, por lo que se ha dejado el código en local. 

\section{Líneas futuras de desarrollo}
En un futuro, se podría continuar con el desarrollo de los siguientes aspectos: 
\begin{itemize}
\item Almacenamiento del código de los sistemas de recomendación en Cloud para permitir implementar la interfaz con diferentes lenguajes de programación. 
\item Desarrollo de diferentes algoritmos para los sistemas de recomendación, para ofrecer mayores puntos de vista al usuario. 
\item Desarrollo del rol administrador con una interfaz diferente para el acceso a la BD, y la modificación, agregación y eliminación de asignaturas. 
\item Desarrollo de una interfaz gráfica en Web. 
\item Desarrollo de  nuevas pestañas en la interfaz, tales como diferencias entre sistemas de recomendación o personalización de perfiles de usuarios o hacer un top de ejecuciones en modelos no deterministas. 
\item Mostrar  las menciones que pertenecen a las  recomendaciones ofrecidas al usuario. 
\item Ampliar a todos los cursos los sistemas de recomendación, no sólo a las asignaturas del cuarto curso. 
\item Personalizar el número de asignaturas de la recomendación, ya que en el cuarto curso no son obligatorias las diez asignaturas ya que un usuario puede no necesitar todas las asignaturas. 
\item Desarrollar una pestaña para priorizar las menciones frente a las recomendaciones, si es ése el interés del alumno. 
\item Aplicar el sistema de recomendación a diferentes grados universitarios, de forma que el sistema no sea excluyente para el grado de Ingeniería Informática. 
\end{itemize}