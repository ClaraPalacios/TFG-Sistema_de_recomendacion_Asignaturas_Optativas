\apendice{Especificación de Requisitos}
Este apéndice se subdivide en los diferentes requisitos necesarios para nuestro proyecto, en su realización y subdivisión. 
\section{Objetivos Generales}
\begin{enumerate}
\item Construcción de un cuestionario anónimo para la recogida de datos iniciales para ser utilizados en el entrenamiento de los sistemas de recomendación. 
\item Desarrollo de diferentes sistemas de recomendación, su entrenamiento en base a los datos recogidos y la devolución de las ponderaciones de las asignaturas no cursadas por un usuario. 
\item Desarrollo de una interfaz gráfica modo usuario para una mejor utilización de los mismos, de forma que las recomendaciones mostradas por el sistema de recomendación sean específicas para determinado usuario. 
\item Desarrollo de una interfaz gráfica modo administrador para la modificación, agregación o eliminación de diferentes calificaciones. 
\end{enumerate}
\section{Catálogo de Requisitos}
\begin{itemize}
\item RF-1. Recogida de datos\\ 
\begin{itemize}
\item RF-1.1. Creación de cuestionario anónimo.\\ 
\item RF-1.2. Distribución del cuestionario. \\ 
\item RF-1.3. Recogida de datos y su almacenamiento. 
\end{itemize}

\item RF-2. Desarrollo de los sistemas de recomendación\\ 
\begin{itemize}
\item RF-2.1. Recogida de datos de Drive. 
\item RF-2.2. Tratamiento de los datos. 
\item RF-2.3. Desarrollo del sistema de recomendación. 
\item RF-2.4. Devolución de las calificaciones. \item Guardado de los datos. 
\end{itemize}

\item Desarrollo de una interfaz gráfica \\ \begin{itemize}
\item Construcción de pestaña de inicio sesión.  \item Lectura de datos almacenados. \item Obtención de datos del registro de usuario. \item Generación de recomendación de calificaciones. \item Muestra de gráficos de calificaciones en diferentes asignaturas. \item Acceso a los datos generales en modo administrador. 
\end{itemize}
\end{itemize}

\section{Especificación de Requisitos}
\begin{itemize}
\item Construcción de ventana de inicio de sesión. 
\begin{itemize}
\item Construcción de botón de inicio sesión. 
\item Construcción de áreas para rellenar usuario y contraseña.  
\item Construcción de la funcionalidad para acceder a la Base de Datos. 
\item Construcción de la funcionalidad para validar usuario y contraseña. 
\end{itemize}
\item Construcción de la pestaña inicial para rellenar el cuestionario para ponderar las asignaturas cursadas. 
\begin{itemize}
\item Construcción del botón de selección del Sistema de Recomendación.
\item Construcción de funcionalidad para llamar al sistema de recomendación  seleccionado. 
\item Construcción de la funcionalidad para ejecutar el sistema de recomendación con los datos rellenados. 
\item Construcción de la funcionalidad para mostrar las calificaciones resultantes no cursadas. 
\end{itemize}
\item Construcción de la funcionalidad del guardado y  el tratamiento de los datos. 
\begin{itemize}
\item Construcción de la funcionalidad de recogida de datos de Drive. 
\item Construcción de la funcionalidad del tratamiento de los datos recogidos. 
\item Construcción de la funcionalidad para el guardado de los datos. 
\end{itemize}
\end{itemize}

\section{Diagramas de Casos de Uso}
\subsection{Diagrama General}
El siguiente diagrama corresponde al  caso de uso general, junto con el diagrama extendido. \ref{fig:Diagrama_Caso_Uso_General}
\begin{figure}[h]
\centering
\includegraphics[width=0.90\textwidth]{Diagrama_Caso_Uso_General}
\caption{Diagrama de caso de uso General}
\label{fig:Diagrama_Caso_Uso_General}
\end{figure}

\section{Tablas de Casos de Uso}
\subsection{Primer Caso de Uso}
La primera tabla corresponde al desarrollo del cuestionario anónimo y la recogida de datos para poder trabajar posteriormente con ellos. La siguiente tabla hace referencia al dicho caso de uso. \ref{tab:1}
\begin{table}[]
\caption{Tabla Caso de Uso 1}
\label{tab:1}
\resizebox{\textwidth}{!}{
\begin{tabular}{ lrrr }
\toprule
\textbf{Nombre} &   Recogida de Datos         \\ 
\textbf{Versión} & 1.0  \\ 
\textbf{Requisitos Funcionales}  & \begin{tabular}[c]{@{}l@{}}RF-1\\ RF-1.1\\ RF-1.2\\ RF-1.3\\\end{tabular}                                                                                                                  \\ 
\textbf{Descripcíon de Requisitos} & \begin{tabular}[r]{@{}l@{}}Se obtendrán los datos de forma anónima  para el entrenamiento\\ de los sistemas de recomendación\\\end{tabular}                                                                                                                    \\
\textbf{Precondiciones}     & No tiene \\
\textbf{Postcondiciones}         &  Se almacenarán los datos en una API de Google \\
\textbf{Autor}         & Clara Palacios Rodrigo \\
\textbf{Importancia}        &  Importante \\ \bottomrule
\end{tabular}
}
\end{table}

\subsection{Segundo Caso de Uso}
La segunda tabla corresponde con el desarrollo de los sistemas de recomendación. Para ello, se debe obtener los datos de la API de Drive, tratarlos y desarrollar los diferentes sistemas de recomendación. La siguieten tabla hace referencia a dicho caso de uso. \ref{tab:2}
\begin{table}[]
\caption{Tabla Caso de Uso 2}
\label{tab:2}
\resizebox{\textwidth}{!}{
\begin{tabular}{ lrrr }
\toprule
\textbf{Nombre} &   Desarrollo de los sistemas de recomendación        \\ 
\textbf{Versión} & 1.0  \\ 
\textbf{Requisitos Funcionales}  & \begin{tabular}[c]{@{}l@{}}RF-2\\ RF-2.1\\ RF-2.2\\ RF-2.3\\RF-2.4\end{tabular}                                                                                                                  \\ 
\textbf{Descripcíon de Requisitos} &  \\
\textbf{Precondiciones}     & Disponer de los datos en Drive \\
\textbf{Postcondiciones}         &   \\
\textbf{Autor}         & Clara Palacios Rodrigo \\
\textbf{Importancia}        & Muy Importante \\ \bottomrule
\end{tabular}
}
\end{table}



