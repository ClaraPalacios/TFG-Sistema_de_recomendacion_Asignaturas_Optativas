\capitulo{8}{Asignaturas de Ingeniería Informática}

Las asignaturas del Grado de Ingeniería Informática en Burgos se subdividen en dos categorías dependiendo de su posibilidad de elección o no. 
\begin{itemize}
\item Asignaturas Obligatorias.
\item Asignaturas Optativas. 
\end{itemize}

\section{Asignaturas Obligatorias}
Las asignaturas obligatorias son aquellas que se cursan en los tres primeros años de la carrera. En su conjunto forman 180 créditos, a seis créditos cada asignatura. 
\begin{itemize}
\item Primer Semestre
\begin{enumerate}
\item \underline{Fundamentos Deontológicos y Jurídicos de las TIC} \\Conocimiento y estudio de la legislación española y su aplicación a las Tecnologías de la Información y la Comunicación, así como la responsabilidad ética y moral de las mismas. 
\item \underline{Álgebra Lineal} \\ Conocimiento y aprendizaje de la resolución de diferentes problemas matemáticos de Álgebra Lineal. 
\item \underline{Informática Básica}\\Aprendizaje de los conocimientos básicos del funcionamiento lógico de las computadoras, así como la aplicación de la Matemática en los mismos.
\item \underline{Fundamentos físicos de la Informática}\\Conocimiento básico de resolución de problemas físicos (electromagnetismo, circuitos eléctricos, principio físico de semiconductores y dispositivos electrónicos)
\item \underline{Matemática discreta}\\Conocimiento de conceptos básicos de la lógica matemática y  algoritmia, así como la resolución de los problemas aplicados a la Informática. 
\end{enumerate}
\end{itemize}

\begin{itemize}
\item Segundo Semestre
\begin{enumerate}[resume]
\item \underline{Inglés Aplicado a la Informática}\\Aprendizaje del vocabulario, expresiones y síntesis de la lengua Inglesa orientada a la informática. 
\item \underline{Cálculo}\\Aprendizaje y resolución de problemas orientados a  los conceptos básicos de la matemática, lógica, algorítmica.
\item \underline{Programación}\\Aprendizaje del lenguaje de programación "C" así como la introducción a la programación estructurada. 
\item \underline{Fundamentos de los Computadores}\\Aprendizaje del funcionamiento básico de los computadores y la programación de los mismos en lenguaje ensamblador. 
\item \underline{Sistemas Operativos}\\Aprendizaje de la arquitectura informática centralizada o distribuida, gestión, desarrollo y mantenimiento de sistemas informáticos.
\end{enumerate}
\end{itemize}
La siguiente tabla corresponde a las asignaturas del Primer Curso~\ref{tab:1}
\begin{table}[]
\centering
\caption{Tabla de asignaturas del Primer Curso Académico}
\label{tab:1}
\rowcolors {2}{gray!35}{}
\resizebox{\textwidth}{!}{
\begin{tabular}{ lrrr }
\toprule
Example                & Orientación & Programación? & Lenguaje de Programación  \\ \midrule
 \parbox{15em}{Fundamentos Deontológicos y Jurídicos de las TIC\\}  & Derecho         & No           & -                \\ 
Álgebra Lineal & Matemática         & No            & -  \\ 
Informática Básica  & Ofimática        & Sí & JavaScript                  \\ 
Fundamentos físicos de la Informática        & Física    & No   & -                \\
Matemática discreta         & Matemática     & No      & -               \\
Inglés Aplicado a la Informática         & Inglés     &No      & -                \\
Cálculo         & Matemática     &No      & -          \\
Programación         & Programación estructurada   &Sí       & C               \\
Fundamentos de los Computadores         & Programación en ensamblador    &Sí       & Ensamblador             \\
Sistemas Operativos         & Uso avanzado de equipos  &Sí           & Bash              \\ \bottomrule

\end{tabular}
}
\end{table}

\begin{itemize}
\item Tercer Semestre
\begin{enumerate}
\item \underline{Metodología de la Programación}\\Aprendizaje de diferentes conceptos tales como modularidad, lenguaje orientado a objetos, conceptos estáticos y dinámicos tales como clases, objetos, herencia, genericidad y resolución de problemas acerca de la robustez de un programa. 
\item \underline{Estadística}\\Aprendizaje de  medición y  la resolución de problemas matemáticos orientados al álgebra lineal, cálculo diferencial e integral estadística y optimización.
\item \underline{Ingeniería del Software}\\Aprendizaje de la metodología del desarrollo sofware, resolución del modelaje de funciones, utilización de la herramienta CASE y comprensión de la teoría de validación sofware, aplicándolo al desarrollo de aplicaciones. 
\item \underline{Bases de Datos}\\Aprendizaje de la gestión  y procesamiento del almacenamiento y acceso a los Sistemas de Información, incluyendo los sistemas basados en Web. 
\item \underline{Arquitectura de Computadores}\\Aprendizaje del funcionamiento de una computadora, comprendiendo las unidades funcionales básicas de los computadores, así como el rendimiento de dichas unidades. 
\end{enumerate}
\end{itemize}
\begin{itemize}
\item Cuarto Semestre
\begin{enumerate}[resume]
\item \underline{Estructuras de Datos}\\Aprendizaje del diseño, desarrollo y evaluación de los sistemas informáticos, calculando la eficiencia algorítmica del acceso a los datos, diferenciando los tipos abstractos de datos y diferenciando el tipo de datos con su abstracción. 
\item \underline{Redes}\\Aprendizaje de la configuración de redes locales, VLANs, enrutamiento IPv4 y IPv6, los diferentes protocolos de transporte de la información y la configuración de servicios de red (DHCP,DNS, SMTP, POP Y HTTP)
\item \underline{Interacción Hombre-Máquina}\\Aprendizaje de las técnicas, bases y soluciones de desarrollo de interfaces teniendo en cuenta la usabilidad ofrecida. 
\item \underline{Fundamentos de Organización y Gestión de Empresas}\\Aprendizaje de la economía de la empresa, la naturaleza, el propósito, la problemática y el desarrollo de la misma, centrándose en las finanzas y el marketing, función directiva y de recursos humanos. 
\item \underline{Análisis y Diseño de Sistemas}\\Aprendizaje del desarrollo de los sistemas software, aportando calidad en los diferentes requisitos, subdividiendo el aprendizaje de la gestión de proyectos Sofware en Dirección, organización, planificación y gestión de proyectos informáticos  de forma que el alumno sea capaz de seleccionar y aplicar los patrones de diseño en la construcción de la aplicación.
\end{enumerate}

\end{itemize}
La siguiente tabla corresponde a las asignaturas del Segundo Curso~\ref{tab:2}
\begin{table}[]
\centering
\caption{Tabla de asignaturas del Segundo Curso Académico}
\label{tab:2}
\rowcolors {2}{gray!35}{}
\resizebox{\textwidth}{!}{
\begin{tabular}{ lrrr }
\toprule
Example                & Orientación & Programación? & Lenguaje de Programación  \\ \midrule
Metodología de la Programación & Informática         & Si           & Java                \\ 
Estadística & Matemática         & No            & -  \\ 
Ingeniería del Software  & Informática        & No & -                  \\ 
Bases de Datos       & Informática    & Sí   & SQL                \\
Arquitectura de Computadores        & Informática     & Sí      &  C++, Cuda              \\
Estructuras de Datos         & Informática     &Sí      & Java                \\
Redes         & Informática     &Sí      & Matlab          \\
Interacción Hombre-Máquina        & Informática   &Sí       & C\#, PHP               \\
Fundamentos de Organización y Gestión de Empresas        & Empresa &No       & - \\
Análisis y Diseño de Sistemas         & Informática  &No           & -              \\ \bottomrule

\end{tabular}
}
\end{table}

\begin{itemize}
\item Quinto Semestre
\begin{enumerate}
\item \underline{Arquitecturas Paralelas}\\Aprendizaje del diseño y construcción de técnicas de programación paralela orientadas a aplicaciones de programación paralela, arquitecturas multinúcleo, memoria compartida y distribuida. 
\item \underline{Sistemas Inteligentes}\\Aprendizaje del lenguaje lógico de proposiciones y diferentes modelos y algoritmos de búsqueda para la resolución de  problemas basados en espacios de estados.
\item \underline{Gestión de Proyectos}\\Aprendizaje de la dirección, organización, gestión y control de proyectos, riesgos y recursos humanos. 
\item \underline{Diseño y Administración de Sistemas y Redes}\\Aprendizaje de la administración de servicios y recursos en sistemas operativos y redes de computadores, optimizando los sistemas, analizando la calidad de los mismos. 
\item \underline{Procesadores del Lenguaje}\\Aprendizaje de atributos (heredados y sintetizados) orientados a la sintaxis XML y  diseño de compiladores (Analizadores sintácticos ascendentes y descendentes).
\end{enumerate}
\end{itemize}
\begin{itemize}
\item Sexto Semestre
\begin{enumerate}[resume]
\item \underline{Programación Concurrente}\\Aprendizaje y diseño de aplicaciones  con diferentes técnicas de sincronización, tiempo real y aplicación de los problemas de programación concurrente a dichas aplicaciones. 
\item \underline{Seguridad Informática}\\Aprendizaje de la problemática de  la seguridad en los diferentes sistemas, su detección y  resolución. Además, se estudiarán los planes de contingencia, su estructura y el ámbito legal en el que se relacionan. 
\item \underline{Aplicaciones de Bases de Datos}\\Aprendizaje de las arquitecturas de las bases de datos, las API para integrar comandos de SQL, transacciones y diferentes niveles de aislamiento y aplicar dichos conocimientos en  un lenguaje de programación conocido (Java).
\item \underline{Algoritmia}\\Aprendizaje y resolución de problemas de complejidad de diferentes algoritmos (incluyendo algoritmos recursivos), diferenciar y seleccionar el algoritmo que mejor convenga para un problema determinado e implementar el esquema junto con el propio algoritmo. 
\item \underline{Métodos Numéricos y Optimización}\\Aprendizaje y resolución de problemas de optimización numérica (discreta y contínua) así como la resolución de ecuaciones, sistemas lineales, interpolación global y segmentaria. 
\end{enumerate}
\end{itemize}
La siguiente tabla corresponde a las asignaturas del Tercer Curso~\ref{tab:3}
\begin{table}[]
\centering
\caption{Tabla de asignaturas del Tercer Curso Académico}
\label{tab:3}
\rowcolors {2}{gray!35}{}
\resizebox{\textwidth}{!}{
\begin{tabular}{ lrrr }
\toprule
Example                & Orientación & Programación? & Lenguaje de Programación  \\ \midrule
Arquitecturas Paralelas & Informática         & Si           & C++                \\ 
Sistemas Inteligentes & Informática         & Sí            & Python  \\ 
Gestión de Proyectos  & Informática        & No & -                  \\ 
Diseño y Administración de Sistemas y Redes       & Informática    & Sí   & Bash               \\
Procesadores del Lenguaje       & Informática     & Sí      &          Flex, Bison \\
Programación Concurrente         & Informática     &Sí      & Java, Ada                \\
Seguridad Informática         & Informática     &No      & -          \\
Aplicaciones de Bases de Datos        & Informática   &Sí       & Java, SQL               \\
Algoritmia        & Informática		 &Sí       & Python \\
Métodos Numéricos y Optimización         & Informática  &Sí           & Matlab              \\ \bottomrule

\end{tabular}
}
\end{table}
\section{Asignaturas Optativas}
Las asignaturas optativas son todas aquellas que el alumno tiene posibilidad de elección en el Cuarto curso académico. Constan de un máximo de 48 créditos, pudiendo   conmutar dos asignaturas por Prácticas curriculares. 
\begin{itemize}
\item Séptimo Semestre
\begin{enumerate}
\item \underline{Diseño e Implementación de Sistemas Digitales}\\Aprendizaje y diseño de la programación de circuitos electrónicos digitales aplicando el diseño a un microprocesador sencillo. 
\item \underline{Gestión de la Información}\\Aprendizaje y desarrollo de los sistemas de información, aplicar pequeñas aplicaciones de Big Data (sistemas de recomendación) y diferentes marcos de problemas de gestión de información. 
\item \underline{Diseño y Mantenimiento del Software}\\Aprendizaje y desarrollo de diferentes patrones de diseño orientados a la aplicación sofware evaluando en qué momento es necesaria su aplicación. 
\item \underline{Organización y Gestión de Empresas}\\Aprendizaje de los principios de los mercados centrándose en el área cinanciera y  los subsistemas productivos. 
\item \underline{Mantenimiento de Equipos Informáticos}\\Aprendizaje de los diferentes elementos de una computadora, su misión y su funcionamiento,  así como el cálculo de su rendimiento, calidad, disponibilidad, fiabilidad y escalabilidad. 
\item \underline{Hardware de Aplicación Específica }\\Aprendizaje del tratamiento de imágenes y vídeos utilizando Matlab.
\item \underline{Control por Computador}\\Aprendizaje, diseño e implementación de los diferentes sistemas de control y la comprensión de los sistemas dinámicos.
\item \underline{Validación y Pruebas}\\Aprendizaje del desarrollo, mantenimiento, evaluación y solución de problemas de servicios y sistemas software, evaluando la usabilidad, ergonomía, accesibilidad y seguridad de los sistemas. 
\item \underline{Computación Neuronal y Evolutiva}\\Aprendizaje de las diferentes redes neuronales, su evolución y aplicación, además de la programación y el entrenamietno de las mismas. Aprendizaje de los algoritmos evolutivos, sus variantes y modelos empleados en la actualidad. 
\item \underline{Programación de Sistemas Operativos}\\Aprendizaje de la estructura de un sistema operativo, su funcionamiento y el análisis para mejorar su eficiencia lo máximo posible orientado a la multitarea, multiproceso y el tiempo de implantación del mismo. 
\end{enumerate}
\end{itemize}
\begin{itemize}
\item Octavo Semestre
\begin{enumerate}
\item \underline{Sistemas Distribuidos}\\Aprendizaje de la estructura y funcionalidades de los sistemas distribuidos e implementar soluciones a diferentes problemas de los sistemas distribuidos en el ámbito de Internet. 
\item \underline{Sistemas Empotrados y de Tiempo Real}\\Aprendizaje de la arquitectura hadware, su funcionamiento, la programación de los mismos  de forma que se permita programar algoritmos de control digital y la programación de microcontroladores. 
\item \underline{Métodos Formales}\\Aprendizaje de los fundamentos de métodos formales, la estandarización de los lenguajes y la obtención de código ejecutable a partir de las especificaciones. 
\item \underline{Nuevas Tecnologías y Empresa}\\Aprendizaje del cambio de modelo en el negocio actual en base a las nuevas tecnologías, y la relación de las mismas con el intercambio de información, productos y servicios con las empresas y personas. 
\item \underline{Minería de Datos}\\Aprendizaje de los conocimientos de la minería de datos e implementar métodos   para el análisis de datos. 
\item \underline{Desarrollo Avanzado de Sistemas Sofware}\\aprendizaje y de los diferentes estándares para la gestión de calidad del sofware, definiendo métricas para su evaluación,  y reestructurar programas mediante la refactorización. \nocite{ubu:docencia}
\end{enumerate}
\end{itemize}
La siguiente tabla corresponde a las asignaturas del Cuarto Curso~\ref{tab:4}
\begin{table}[]
\caption{Tabla de asignaturas del Cuarto Curso Académico}
\label{tab:4}
\rowcolors {2}{gray!35}{}
\resizebox{\textwidth}{!}{
\begin{tabular}{ lrrr }
\toprule
Example                & Orientación & Programación? & Lenguaje de Programación  \\ \midrule
Diseño e Implementación de Sistemas Digitales & Informática         & Si           & HDL                \\ 
Gestión de la Información & Informática         & Sí            & Python  \\ 
Diseño y Mantenimiento del Software  & Informática        & Sí & Java                  \\ 
Organización y Gestión de Empresas       & Empresa    & Sí   & Matlab               \\
Mantenimiento de Equipos Informáticos       & Informática     & No      &- \\
Hardware de Aplicación Específica         & Informática     &Sí      &    Matlab          \\
Control por Computador         & Informática     &No      &          \\
Validación y Pruebas        & Informática   &Sí       &                \\
Computación Neuronal y Evolutiva        & Informática		 &Sí       & Matlab, Java\\
Programación de Sistemas Operativos        & Informática		 &Sí       & C++, C \\
Sistemas Distribuidos        & Informática		 &Sí       &  Java\\
Sistemas Empotrados y de Tiempo Real        & Informática		 &Sí       &  \\
Métodos Formales        & Informática		 &Sí       &  \\
Nuevas Tecnologías y Empresa        & Informática		 &Sí       & Python \\
Minería de Datos        & Informática		 &Sí       &  Java, Python\\
Desarrollo Avanzado de Sistemas Sofware        & Informática		 &Sí       & Java   \\ \bottomrule

\end{tabular}
}
\end{table}


