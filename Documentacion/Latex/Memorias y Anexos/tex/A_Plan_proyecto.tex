\apendice{Planificación}

\section{Introducción}
En el desarrollo de este proyecto, utilizaremos la metodología SCRUM, con un desarrollo incremental con una duración de 2 semanas por Sprint. 
La organización en GitHub se realizará del siguiente modo: 
\begin{itemize}
\item Creación de un nuevo Milestone con una duración de 2 semanas el día de la reunión. 
\item Creación de los issues básicos necesarios para dicho Milestone. 
\item Desarrollo de los issues y la creación de los nuevos issues necesarios. 
\item Utilización de la herramienta Zenhub para el seguimiento de las tareas. 
\item Cierre de las issues una vez finalizadas para observar el avance de las tareas de forma real frente al progreso ideal. 
\end{itemize}

\section{Planificación temporal}
La evolución bisemanal de las tareas se ha realizado de la siguiente manera: 
\subsection{\textbf{Sprint 1}  (15/01/2018-29/01/2018) }
El primer Sprint, orientado hacia la explicación del desarrollo del proyecto. Se decidirán las herramientas básicas de la gestión de tareas, documentación de memoria y anexos y las referencias bibliográficas. 
Por ello: 
\begin{itemize}
\item Se ha documentado y probado la utilización de \LaTeX como editor de texto.
\item Se han documentado y probado los gestores de versiones de metodología ágil. 
\item 

\end{itemize}

