\documentclass[a4paper,11pt,oneside]{memoir}

% Castellano
\usepackage[spanish]{babel}
\selectlanguage{spanish}
\usepackage[utf8]{inputenc}
\usepackage{placeins}
\usepackage[T1]{fontenc}
\usepackage{caption}
\usepackage{subcaption}
\usepackage[linesnumbered,ruled,vlined,spanish]{algorithm2e}
\usepackage{enumitem}
\usepackage{chngcntr}
\RequirePackage{booktabs}
\RequirePackage[table,xcdraw]{xcolor}
\RequirePackage{xtab}
\RequirePackage{multirow}
\usepackage{amsmath}
\usepackage{hyperref}
\counterwithout{footnote}{chapter}
% Links
\hypersetup{
	allcolors = {red},
	urlcolor ={black}%solo para la version de imprimir
}

% Ecuaciones
\usepackage{amsmath}

% Rutas de fichero / paquete
\newcommand{\ruta}[1]{{\sffamily #1}}

% Párrafos
\nonzeroparskip


% Imagenes
\usepackage{graphicx}
\newcommand{\imagen}[2]{
	\begin{figure}[!h]
		\centering
		\includegraphics[width=0.9\textwidth]{#1}
		\caption{#2}\label{fig:#1}
	\end{figure}
	\FloatBarrier
}

\newcommand{\imagenflotante}[2]{
	\begin{figure}%[!h]
		\centering
		\includegraphics[width=0.9\textwidth]{#1}
		\caption{#2}\label{fig:#1}
	\end{figure}
}



% El comando \figura nos permite insertar figuras comodamente, y utilizando
% siempre el mismo formato. Los parametros son:
% 1 -> Porcentaje del ancho de página que ocupará la figura (de 0 a 1)
% 2 --> Fichero de la imagen
% 3 --> Texto a pie de imagen
% 4 --> Etiqueta (label) para referencias
% 5 --> Opciones que queramos pasarle al \includegraphics
% 6 --> Opciones de posicionamiento a pasarle a \begin{figure}
\newcommand{\figuraConPosicion}[6]{%
  \setlength{\anchoFloat}{#1\textwidth}%
  \addtolength{\anchoFloat}{-4\fboxsep}%
  \setlength{\anchoFigura}{\anchoFloat}%
  \begin{figure}[#6]
    \begin{center}%
      \Ovalbox{%
        \begin{minipage}{\anchoFloat}%
          \begin{center}%
            \includegraphics[width=\anchoFigura,#5]{#2}%
            \caption{#3}%
            \label{#4}%
          \end{center}%
        \end{minipage}
      }%
    \end{center}%
  \end{figure}%
}

%
% Comando para incluir imágenes en formato apaisado (sin marco).
\newcommand{\figuraApaisadaSinMarco}[5]{%
  \begin{figure}%
    \begin{center}%
    \includegraphics[angle=90,height=#1\textheight,#5]{#2}%
    \caption{#3}%
    \label{#4}%
    \end{center}%
  \end{figure}%
}
% Para las tablas
\newcommand{\otoprule}{\midrule [\heavyrulewidth]}
%
% Nuevo comando para tablas pequeñas (menos de una página).
\newcommand{\tablaSmall}[5]{%
 \begin{table}
  \begin{center}
   \rowcolors {2}{gray!35}{}
   \begin{tabular}{#2}
    \toprule
    #4
    \otoprule
    #5
    \bottomrule
   \end{tabular}
   \caption{#1}
   \label{tabla:#3}
  \end{center}
 \end{table}
}

%
% Nuevo comando para tablas pequeñas (menos de una página).
\newcommand{\tablaSmallSinColores}[5]{%
 \begin{table}[H]
  \begin{center}
   \begin{tabular}{#2}
    \toprule
    #4
    \otoprule
    #5
    \bottomrule
   \end{tabular}
   \caption{#1}
   \label{tabla:#3}
  \end{center}
 \end{table}
}

\newcommand{\tablaApaisadaSmall}[5]{%
\begin{landscape}
  \begin{table}
   \begin{center}
    \rowcolors {2}{gray!35}{}
    \begin{tabular}{#2}
     \toprule
     #4
     \otoprule
     #5
     \bottomrule
    \end{tabular}
    \caption{#1}
    \label{tabla:#3}
   \end{center}
  \end{table}
\end{landscape}
}

%
% Nuevo comando para tablas grandes con cabecera y filas alternas coloreadas en gris.
\newcommand{\tabla}[6]{%
  \begin{center}
    \tablefirsthead{
      \toprule
      #5
      \otoprule
    }
    \tablehead{
      \multicolumn{#3}{l}{\small\sl continúa desde la página anterior}\\
      \toprule
      #5
      \otoprule
    }
    \tabletail{
      \hline
      \multicolumn{#3}{r}{\small\sl continúa en la página siguiente}\\
    }
    \tablelasttail{
      \hline
    }
    \bottomcaption{#1}
    \rowcolors {2}{gray!35}{}
    \begin{xtabular}{#2}
      #6
      \bottomrule
    \end{xtabular}
    \label{tabla:#4}
  \end{center}
}

%
% Nuevo comando para tablas grandes con cabecera.
\newcommand{\tablaSinColores}[6]{%
  \begin{center}
    \tablefirsthead{
      \toprule
      #5
      \otoprule
    }
    \tablehead{
      \multicolumn{#3}{l}{\small\sl continúa desde la página anterior}\\
      \toprule
      #5
      \otoprule
    }
    \tabletail{
      \hline
      \multicolumn{#3}{r}{\small\sl continúa en la página siguiente}\\
    }
    \tablelasttail{
      \hline
    }
    \bottomcaption{#1}
    \begin{xtabular}{#2}
      #6
      \bottomrule
    \end{xtabular}
    \label{tabla:#4}
  \end{center}
}

%
% Nuevo comando para tablas grandes sin cabecera.
\newcommand{\tablaSinCabecera}[5]{%
  \begin{center}
    \tablefirsthead{
      \toprule
    }
    \tablehead{
      \multicolumn{#3}{l}{\small\sl continúa desde la página anterior}\\
      \hline
    }
    \tabletail{
      \hline
      \multicolumn{#3}{r}{\small\sl continúa en la página siguiente}\\
    }
    \tablelasttail{
      \hline
    }
    \bottomcaption{#1}
  \begin{xtabular}{#2}
    #5
   \bottomrule
  \end{xtabular}
  \label{tabla:#4}
  \end{center}
}



\definecolor{cgoLight}{HTML}{EEEEEE}
\definecolor{cgoExtralight}{HTML}{FFFFFF}

%
% Nuevo comando para tablas grandes sin cabecera.
\newcommand{\tablaSinCabeceraConBandas}[5]{%
  \begin{center}
    \tablefirsthead{
      \toprule
    }
    \tablehead{
      \multicolumn{#3}{l}{\small\sl continúa desde la página anterior}\\
      \hline
    }
    \tabletail{
      \hline
      \multicolumn{#3}{r}{\small\sl continúa en la página siguiente}\\
    }
    \tablelasttail{
      \hline
    }
    \bottomcaption{#1}
    \rowcolors[]{1}{cgoExtralight}{cgoLight}

  \begin{xtabular}{#2}
    #5
   \bottomrule
  \end{xtabular}
  \label{tabla:#4}
  \end{center}
}


















\graphicspath{ {./img/} }

% Capítulos
\chapterstyle{bianchi}
\newcommand{\capitulo}[2]{
	\setcounter{chapter}{#1}
	\setcounter{section}{0}
	\chapter*{#2}
	\addcontentsline{toc}{chapter}{#2}
	\markboth{#2}{#2}
}

% Apéndices
\renewcommand{\appendixname}{Apéndice}
\renewcommand*\cftappendixname{\appendixname}

\newcommand{\apendice}[1]{
	%\renewcommand{\thechapter}{A}
	\chapter{#1}
}

\renewcommand*\cftappendixname{\appendixname\ }

% Formato de portada
\makeatletter
\usepackage{xcolor}
\newcommand{\tutor}[1]{\def\@tutor{#1}}
\newcommand{\course}[1]{\def\@course{#1}}
\definecolor{cpardoBox}{HTML}{E6E6FF}
\def\maketitle{
  \null
  \thispagestyle{empty}
  % Cabecera ----------------
\noindent\includegraphics[width=\textwidth]{cabecera}\vspace{1cm}%
  \vfill
  % Título proyecto y escudo informática ----------------
  \colorbox{cpardoBox}{%
    \begin{minipage}{.8\textwidth}
      \vspace{.5cm}\Large
      \begin{center}
      \textbf{TFG del Grado en Ingeniería Informática}\vspace{.6cm}\\
      \textbf{\LARGE\@title{}}
      \end{center}
      \vspace{.2cm}
    \end{minipage}

  }%
  \hfill\begin{minipage}{.20\textwidth}
    \includegraphics[width=\textwidth]{escudoInfor}
  \end{minipage}
  \vfill
  % Datos de alumno, curso y tutores ------------------
  \begin{center}%
  {%
    \noindent\LARGE
    Presentado por \@author{}\\ 
    en Universidad de Burgos --- \@date{}\\
    Tutor: \@tutor{}\\
  }%
  \end{center}%
  \null
  \cleardoublepage
  }
\makeatother

\newcommand{\nombre}{Clara Palacios Rodrigo} %%% cambio de comando

% Datos de portada
\title{Sistema de Recomendación de Asignaturas Optativas}
\author{\nombre}
\tutor{Dr. José Ignacio Santos Martín, Virginia Ahedo García}
\date{\today}

\begin{document}

\maketitle



\null\cleardoublepage


%%%%%%%%%%%%%%%%%%%%%%%%%%%%%%%%%%%%%%%%%%%%%%%%%%%%%%%%%%%%%%%%%%%%%%%%%%%%%%%%%%%%%%%%
\pagestyle{empty}


\noindent\includegraphics[width=\textwidth]{cabecera}\vspace{1cm}

\noindent Dr. José Ignacio Santos Martín y Dña. Virginia Ahedo García, profesores del departamento de Ingeniería Civil , área de Organización de Empresas.
		

\noindent Expone:

\noindent Que el alumno D. \nombre, con DNI 71307844-R, ha realizado el Trabajo final de Grado en Ingeniería Informática titulado Sistema de Recomendación de Asignaturas Optativas . 

\noindent Y que dicho trabajo ha sido realizado por el alumno bajo la dirección del que suscribe, en virtud de lo cual se autoriza su presentación y defensa.

\begin{center} %\large
En Burgos, {\large \today}
\end{center}

\vfill\vfill\vfill

% Author and supervisor
\begin{minipage}{0.45\textwidth}
\begin{flushleft} %\large
Vº. Bº. del Tutor:\\[2cm]
Dr. José Ignacio Santos Martín
\end{flushleft}
\end{minipage}
\hfill
\begin{minipage}{0.45\textwidth}
\begin{flushleft} %\large
Vº. Bº. del Tutor:\\[2cm]
Dña. Virginia Ahedo García
\end{flushleft}
\end{minipage}
\hfill

\vfill

% para casos con solo un tutor comentar lo anterior
% y descomentar lo siguiente
%Vº. Bº. del Tutor:\\[2cm]
%D. nombre tutor


\null\cleardoublepage
\null\cleardoublepage




\frontmatter

% Abstract en castellano
\renewcommand*\abstractname{Resumen}
\begin{abstract}
Los sistemas de recomendación son una herramienta informática muy utilizada en una gran variedad de problemas donde se persigue escoger un subconjunto de items dentro de un catálogo de productos que responda adecuadamente a las preferencias de un usuario. Ejemplos importantes los encontramos en las empresas de comercio electrónico que necesitan proponer a sus clientes aquellos productos potencialmente interesantes para ellos. La funcionalidad de un sistema de recomendación  puede utilizar diferentes técnicas y algoritmos para resolver este problema. De entre las diferentes técnicas destacan los filtros colaborativos que utilizan como datos únicamente una matriz de utilidad que expresa las valoraciones que los usuarios han dado a los diferentes productos consumidos. Los filtros colaborativos funcionan bastante bien en aquellos problemas en los que se dispone de una gran cantidad de productos y de usuarios. En este Trabajo Final de Grado se ha decidido diseñar e implementar un sistema de recomendación de filtro colaborativo que permite ofrecer recomendaciones de asignaturas optativas de cuarto curso a los alumnos del Grado en Ingeniería Informática de la Universidad de Burgos.
\end{abstract}

\renewcommand*\abstractname{Descriptores}
\begin{abstract}
Python, Recomendación, Predicción, Formalización, Filtro Colaborativo (FC).
\end{abstract}

\clearpage

% Abstract en inglés
\renewcommand*\abstractname{Abstract}
\begin{abstract}
Recommender Systems are a computer tool used in a wide variety of problems where we need to choose a subset of items within a catalog of products that responds appropriately to the preferences of a user. We see interesting examples in e-commerce where firms need to offer their customers those products that are potentially interesting for them. The functionality of a Recommender System uses different techniques and algorithms to solve the election problem. Among the different techniques, we find the collaborative filtering that only requires a utility matrix that represents the ratings that users have given to the products consumed. Collaborative filtering works quite well in those problems where there is a large number of products and users. In this final project, we have designed and implemented a collaborative filtering that offers recommendations of optional subjects to 4th-year students of the Computer Science Degree of the University of Burgos.

\end{abstract}

\renewcommand*\abstractname{Keywords}
\begin{abstract}

Python, Recommendation, Prediction, Formalization, Collaborative Filter (CF) .\end{abstract}

\clearpage

% Indices
\tableofcontents

\clearpage

\listoffigures

\clearpage

\listoftables

%\clearpage

\mainmatter
\newcommand{\grad}{$^{\circ}$}
\capitulo{1}{Introducción}
%\maketitle
\nonzeroparskip
Este proyecto consiste en resolver el problema de la elección de las asignaturas optativas del último curso entre los estudiantes del grado de Ingeniería Informática. \\

Dado el desconocimiento de la materia impartida en 4º curso y la falta de visión de la relación entre la misma, un gran porcentaje de estudiantes universitarios consideran que su elección en alguna de las asignaturas no ha sido la correcta. \\

Por ese motivo, para evitar confusiones por parte de los alumnos, se creará una aplicación mediante la cual, el usuario rellenará un  cuestionario anónimo, introduciendo  una calificación por cada una de las asignaturas cursadas, de forma que el sistema sea capaz de indicarle cuál de las optativas se ajusta más a su criterio. 

Sin embargo, a pesar de poner en práctica diferentes filtros colaborativos, los resultados pueden mejorarse, ya que el número de datos recopilados hasta el momento son mínimos con respecto a los sistemas de recomendación utilizados en grandes empresas, de forma que la recomendación ofrecida al usuario es simplemente orientativa. 
\capitulo{2}{Objetivos del proyecto}
Los objetivos del proyecto se basan en la resolución de la incógnita de qué asignaturas optativas escoger en el último curso académico de Ingeniería Informática basándose en las preferencias de los cursos anteriores. Para ello se debe: 
\begin{itemize}
\item Analizar el problema propuesto y escoger el algoritmo del sistema de recomendación preferente: 
\begin{itemize}
\item Documentarse en los diferentes Sistemas de Recomendación existentes y la decisión de qué sistema se ajusta más a este problema. 
\item Documentarse en diferentes métodos de creación de Cuestionarios Anónimos así como la recogida de los datos. 
\end{itemize}
\item Creación de un cuestionario y la obtención de los datos. 
\begin{itemize}
\item Difusión entre  los alumnos que estén cursando -o hayan finalizado la carrera- del cuestionario para la obtención de los datos de sus preferencias en las asignaturas. 
\item Tratamiento de dichos datos con el sistema de recomendación escogido.
\end{itemize}
\item Creación del cuestionario final junto con una sencilla interfaz gráfica para la interacción con el usuario. 
\end{itemize}
\capitulo{3}{Conceptos teóricos}
\section{Sistemas de Recomendación}
Un sistema de recomendación es aquella herramienta encargada de recopilar, filtrar y devolver un conjunto de datos destinados para ofrecer al consumidor una orientación en base a sus gustos o en base a los gustos de otros usuarios cuyos aspectos introducidos coincidan.\nocite{wiki:recomendacion} \\
Los sistemas de recomendación se subdividen en dos grupos: 
\subsection{Filtro colaborativo}
Técnica utilizada en los sistemas de recomendación basada en predicción de las preferencias basado en las selecciones de los usuarios. Para ello, el sistema funciona de la siguiente forma: \nocite{andres_gonzalez_sistemas_2014}
\begin{enumerate}
\item El usuario rellena el formulario con respecto a las preferencias de aquellos elementos conocidos.
\item El sistema recopila los datos y realiza las comparaciones con otros usuarios para determinar las similitudes. \nocite{upf:recomendacion}
\item El sistema informa de aquellos items en los que el cliente podría coincidir con el resto de usuarios pero no ha rellenado. \cite{wiki:filtrado}
\end{enumerate}
Una explicación más sencilla sería en que el sistema ofrece a un usuario lo más popular entre aquellos con sus mismos gustos.\nocite{ucm:recomendacion} 
Existen dos clases de Sistemas de Recomendación con filtro colabroativo: 
\subsubsection{Basado en modelos}
La estimación de los items se basa en el rating de los items de los diferentes usuarios. 
Por ello, para su utilización, se crea inicialmente una tabla, en la que las filas son los diferentes usuarios, mientras que las columnas hacen referencia a los diferentes elementos.

\begin{table}[]
\centering
\caption{Tabla de ejemplo de filtro colaborativo basado en modelo}
\label{tab:1}
\rowcolors {2}{gray!35}{}
\begin{tabular}{ lrrrr }
\toprule
Example                & $item_{1}$ & $item_{2}$ & ... & $item_{n}$ \\ \midrule
$user_{1}$ & 5         & 2           & ...           & 4       \\ 
$user_{2}$ & 5         & 1            & ...           & 3       \\ 
... & 2        & 5           & ...           & 5       \\ 
$user_{n}$ & 4         & ?           & ...          & 3       \\ \bottomrule
\end{tabular}
\end{table}

Posteriormente se ajusta el modelo para proponer al usuario un valor recomendado para el item desconocido (?).
\subsubsection{Basado en memoria}
La estimación de los items se basa en los rating del cliente con respecto a los items de los diferentes usuarios. El sistema, tras rellenar la tabla de usuarios/items busca las similitudes entre las preferencias de los usuarios para ofrecer una recomendación. 
A su vez, el sistema de recomendación con filtro colaborativo basado en memoria se puede subdividir en: 
\begin{enumerate}

\item  \textbf{Basado en Usuarios}\\ La estimación del  $item_{i}$ de un $usuario_{u}$ concreto se basa en los rating del $item_{i}$  de los usuarios  con mayor proximidad al $usuario_{u}$.

\item \textbf{Basado en Productos}\\La estimación del  $item_{i}$ de un $usuario_{u}$ concreto se basa en los rating del $usuario_{u}$ a los items similares a $item_{i}$.

\item Híbrido
\end{enumerate}

Nos centraremos en esta técnica.
\subsection{Filtrado basado en contenido}
Técnica utilizada en los sistemas de recomendación,  basada en asignar unos pesos a cada uno de los items conocidos por el usuario, de forma que se pueda utilizar el algoritmo para extraer y recomendar aquellos items que tengan mayor similitud con los gustos del usuario.
Su explicación se basaría en ofrecer al usuario aquello que se relaciona más con lo que le ha gustado. 

\subsection{Filtrado basado en conocimiento}
Técnica utilizada en los sistemas de recomendación, basada en 
\subsection{Filtrado híbrido}


\capitulo{4}{Técnicas y herramientas}

\section{Metodología Ágil}

La metodología ágil es aquella forma de toma de decisiones en los proyectos software basado en un desarrollo iterativo, evaluando las necesidades y tareas necesarias a la par de la realización, para añadir funcionalidad en el proyecto.\cite{wiki:desarrollo}\\ 
\subsection{KANBAN}
Sistema de información basado en la metodología ágil. Utiliza tarjetas para representar la información de forma visual, mejorando la distribución de trabajo y la organización del mismo. \cite{ kanban:metodo}
La representación Kanban se desarrolla en un tablero, asignando tareas a los diferentes miembros, utilizando tarjetas para indicar las diferentes etapas en las que se encuentran las subtareas del proyecto. \cite{wiki:kanban}
 
\subsection{SCRUM}
Modelo de referencia basado en metodología ágil basado en un desarrollo incremental que permite modificar tareas e ideas una vez comenzado el proyecto. \nocite{agile:scrum}
Hemos seleccionado esta opción, ya que entre sus ventajas podemos indicar: 
\begin{itemize}
\item Flexibilidad ante las diferentes necesidades y posibles cambios. \nocite{proyectos:scrum}
\item Entrega progresiva del proyecto. 
\item Control en todas las etapas del proyecto. 
\end{itemize}
\section{Repositorio de código y control de versiones}
\subsection{VersionOne}
\nonzeroparskip
Plataforma compacta basada en gestión ágil, está estructurada para admitir diferentes metodolgías ágiles, tales como Scrum, Kanban, Lean o XP, permitiendo la rápida escalabilidad de los proyectos. \cite{versionOne:soporte} \\
Como ventajas podemos indicar que: 
\begin{itemize}
\item Fácil de utilizar. 
\item Gran cantidad de métricas ágiles 
\item Fácilmente escalable. 
\end{itemize}
Sin embargo, es un software de pago, ya que las utilidades necesarias para el Trabajo Final de Carrera no están incluidas en la versión gratuita por lo que ha sido descartada esta opción. 
\subsection{Bitbucket}
Plataforma de almacenamiento de código que utiliza el sistema de control de versiones \nocite{wiki:bitbucket} basado en gestión ágil.
\begin{itemize}
\item Ventajas
\begin{itemize}
\item Número ilimitado de creación de repositorios  privados gratuitos. 
\item Control de versiones Git y Mercurial
\item Software de comunicación entre los diferentes colaboradores de un proyecto.\nocite{redes:bitbucket} 
\end{itemize}
\end{itemize}
\begin{itemize}
\item Inconvenientes
\begin{itemize}
\item Número limitado de colaboradores en un proyecto gratuito. 
\item  Exclusión de un soporte para el sistema operativo Linux.
\item El coste por tener un mayor número de colaboradores en un proyecto es muy superior a Github-cuyo coste se basa en el número de repositorios privados existentes.\nocite{rocreguant:bitbucket} 
\end{itemize}
\end{itemize}


\subsection{Gitlab}
Sistema de control de versiones basado en gestión ágil \nocite{wiki:gitLab} y gestor de repositorio de código con licencia MIT-de software libre permisivo-\nocite{wiki:MIT}
\begin{itemize}
\item Ventajas
\begin{itemize}
\item Facilidad en adjuntar archivos en los issues.
\item Protección de ramificaciones frente a cambios utilizando el sistema de niveles de autorización. \cite{openbinars:gitlab}
\item Posibilidad de crear repositorios privados de forma gratuita. 
\item No hay limitaciones ni en el número de repositorios privados ni en el número de colaboradores. 
\end{itemize}
\end{itemize}
\begin{itemize}
\item Inconvenientes
\begin{itemize}
\item Menor cantidad de miembros en la comunidad. 
\item Gestor de repositorio totalmente nuevo, por lo que se debería estudiar su funcionamiento. \nocite{platzi:gitlab}
\end{itemize}
\end{itemize}

\subsection{SourceForge}
Sitio web  de sofware libre para almacenar y compartir proyectos de código abierto.
\begin{itemize}
\item Ventajas
\begin{itemize}
\item Gran número de proyectos alojados.
\end{itemize}
\end{itemize}
\begin{itemize}
\item Inconvenientes
\begin{itemize}
\item Monetización de la descarga de los proyectos basándose en la inclusión de anuncios publicitarios en los instaladores.\cite{wiki:SouceForge}
\item Necesidad de aprendizaje 
\end{itemize}
\end{itemize}
\subsection{Github}
GitHub es una plataforma de desarrollo colaborativo de código para alojar proyectos utilizando el sistema de control de versiones Git.\nocite{git:gith} Permite gestionar los cambios de código indicando qué se ha modificado, añadido o eliminado en cualquier commit realizado. \\Entre sus ventajas e inconvenientes, podemos indicar: 
\begin{itemize}
\item Ventajas
\begin{enumerate}
\item Servicio gratuito en los repositorios públicos. 
\item No es necesario realizar copias de seguridad del código, ya que utiliza un sistema de control de versiones. 
\item Compatible con diferentes sistemas operativos (Windows, Linux, OS...). \nocite{gits:gitHub}
\end{enumerate}
\end{itemize}
\begin{itemize}
\item Inconvenientes
\begin{enumerate}
\item Los repositorios privados no son gratuitos. 
\end{enumerate}
\end{itemize}
Sin embargo, a pesar del inconveniente del pago en caso de desear un repositorio privado, las ventajas de la utilización de GitHub tienen un mayor peso, además de haber sido la plataforma utilizada en los cursos académicos del grado de Ingeniería Informática. 
\section{Gestores de Tareas}
\subsection{Trello}
Herramienta para la gestión y seguimiento\nocite{xelso:trello} de las tareas en tiempo real. Para acceder a dicha herramienta, se debe crear una cuenta en Trello.com. Sin embargo, para evitar el uso masivo de herramientas y aplicaciones innecesarias en nuestro proyecto, descartamos esta opción. 

\subsection{Hojas de Cálculo}
Una hoja de cálculo es una herramienta  para la manipulación de datos y gráficos en hojas divididas en celdas. Se podría  Sin embargo, ante la falta de orden y de un correcto seguimiento, se ha descartado dicha herramienta desde un comienzo. 
\subsection{Zenhub}
ZenHub es una extensión integrada en Chrome orientada a la gestión y el control de proyectos utilizando pizarras o columnas en las que se ordenan las issues creadas para mejorar la clasificación de las mismas. 
Utilizaremos esta herramienta para la gestión y el seguimiento de las diferentes tareas, creando nuevas Pipelines  y utilizando las existentes. Entre ellas, podemos encontrar: 
\begin{itemize}
\item ICEBOX \\ Issues que se hayan comenzado, y sin embargo, por cualquier motivo, se deben dejar paradas. 
\item Backlog \\ Issues que puede que  se vayan a desarrollar más adelante.
\item To do \\Isues pendiendtes por desarrollar. 
\item In progress \\Issues que están en desarrollo en ese mismo momento
\item Clossed \\Issues que se han terminado y han sido cerradas. 
\end{itemize}
\section{Editores de Texto}
\subsection{Microsoft Word}
Microsoft Word es un programa de la empresa Microsoft creado para la edición de texto\cite{wiki:Word}
\begin{itemize}
\item Ventajas
\begin{itemize}
\item Interfaz gráfica que permite una mejor comprensión de la herramienta para los usuarios nóveles. \nocite{elblogdel1:Word}
\item Sencillez ante la edición de texto así como la utilización de los diferentes formatos.
\end{itemize}
\end{itemize}
\begin{itemize}
\item Inconvenientes
\begin{itemize}
\item Fallo de seguridad en los complementos la herramienta que permite el robo de archivos mediante la introducción de un documento con código oculto.\cite{monografias:Word}
\item Limitación de la capacidad para la inclusión y el tratamiento de las imágenes.
\item Problemática en el mantenimiento del mismo formato en todo el documento.
\end{itemize}
\end{itemize}
\subsection{OpenOffice}
Paquete de sofware de código abierto destinado para el procesamiento y la edición de texto. 
\begin{itemize}
\item Ventajas
\begin{itemize}
\item Al igual que Microsoft Word, consta de una interfaz gráfica  fácil de utilizar en la edición y tratamiento de texto, gráficos y tablas. 
\item No tiene coste de licencia-al contrario que Microsoft Word.\nocite{apache:office}
\item Multiplataforma. 
\end{itemize}
\end{itemize}
\begin{itemize}
\item Inconvenientes
\begin{itemize}
\item Lentitud en el procesamiento, edición y guardado de archivos\nocite{theclandia:office}
\item Incompatibilidad con Microsoft Word, por lo que varias de las acciones disponibles en OpenOffice Writer no se reconocen en Microsoft Word; acciones tales como es el pegado de una tabla.
\item Al igual que ocurre en Microsoft Word, puede ser dificultoso mantener el mismo formato en todo el documento.
\end{itemize}
\end{itemize}
\subsection{\LaTeX}
Sistema de edición y procesamiento de texto basado en comandos de TeX\footnote{Sistema de tipografía desarrollado en 1985 por Donald E. Knuth, de licencia libre, es utilizado en entornos académicos por el alto número de funcionalidades que ofrece\cite{wiki:Tex}.}\\
Los archivos de \LaTeX tienen la extensión .tex, cuyo contenido debe ser compilado de forma previa a su visualización. En este proyecto, la compilación será PDFLaTeX, para obtener el formato PDF del fichero escrito. 
\begin{itemize}
\item Ventajas
\begin{itemize}
\item Sencillez en la utilización del mismo formato durante el proyecto, referencias cruzadas y numeración (Documento estructurado)
\item Calidad en la edición de texto y funciones matemáticas. 
\item Multiplataforma.
\item Es una herramienta portable y gratuita.
\end{itemize}
\end{itemize}
\begin{itemize}
\item Inconvenientes
\begin{itemize}
\item Es una herramienta compleja, ya que requiere un periodo previo a su utilización de aprendizaje de comandos básicos. 
\item Dificultad en la introducción de imágenes  y bibliografías.
\item Necesidad de compilación por cada cambio realizado para observar los resultados finales. \nocite{aq:LaTex} 
\end{itemize}
\end{itemize}
\LaTeX  será el sistema de edición de texto escogido, ya que, a pesar de la dificultad de su aprendizaje, hemos considerado que tiene mayor peso las ventajas que los inconvenientes propuestos.

\section{Gestores de Referencias Bibliográficas}
\subsection{Mendeley}
Mendeley es un sofware  gratuito destinado a la gestión de referencias bibliográficas. Compatible con diferentes versiones de Microsoft Word, OpenOffice y BibTex tiene como característica principal la sincronización de las referencias tanto en el equipo  propio como en Web. \nocite{ucm:mendeley}
Entre sus ventajas podemos destacar: 
\begin{itemize}
\item Multiplataforma
\item Organización de las referencias. 
\item Compartir referencias bibliográficas entre diferentes usuarios. 

\end{itemize}
\subsection{Zotero}
Software libre de código abierto destinado a la gestión de referencias bibliográficas. \cite{bibl:zotero} Utilizado en Firefox, Chrome, Safari y Opera, sigue cinco principios básicos: 
\begin{enumerate}
\item Recopilar información. 
\item Organizar los recursos en la biblioteca. 
\item Citar las referencias bibliográficas de manera automática en la edición de texto. 
\item Sincronización de la biblioteca en un servidor. 
\item Colaboración al permitir compartir las bibliotecas creadas con el resto de usuarios. 
\end{enumerate}
\begin{itemize}
\item Ventajas
\begin{itemize}
\item Multiplataforma
\item Extracción automática de las citas en la edición de texto. 
\item Sincronización con diferentes editores de texto (Microsoft Word, OpenOffice y LibreOffice)\cite{wiki:Zotero}
\item Importación de metadatos de  un amplio número de soportes.
\item Reconoce bibliotecas en formato BibTex.
\end{itemize}
\end{itemize}
\begin{itemize}
\item Inconvenientes
\begin{itemize}
\item No dispone de lector de PDF. 
\item Necesidad de utilizar aplicaciones de terceros para obtener todas las funcionalidades disponibles. 
\end{itemize}
\end{itemize}
\subsection{BibTex}
\nonzeroparskip
La herramienta elegida para la bibliografía es BibTex, diseñada como una utilidad de apoyo bibliográfico para LateX. \\
\nonzeroparskip
Para utilizarlo, se emplean los ficheros.bib en los que se encuentre la bibliografía necesaria (librerías), de forma que BibTex añadirá a la bibliografía las librerías que hayan sido citadas en el documento. \\
\nonzeroparskip
Además, para obtener los datos de la bibliografia, basta con abrir una pestaña de  "school.google.es, marcar en Configuración la opción "Mostrar enlaces para importar citas a BibTex". De esta forma, una vez encontrada la página deseada, tan sólo es necesario copiar la bibliografía que se muestre y guardarla en nuestra librería.\cite{perez1968titulo} 

\section{Recogida de datos}
La recogida de datos se ha realizado mediante un cuestionario anónimo difundido entre los Alumnos que hayan cursado la carrera del Grado de Ingeniería Informática en Burgos. Entre las diferentes opciones que existían destacamos: 
\subsection{SurveyMonkey}
Herramienta de creación de cuestionarios online con versión gratuita para realizar: 
\begin{itemize}
\item 15 tipos de preguntas diferentes. 
\item 10 preguntas máximo por cuestionario. 
\item Un máximo de 100 respuestas.  
\end{itemize}
Hemos descartado dicha herramienta, ya que en la versión gratuita no permite la exportación de datos en ningún soporte, además de la limitación del número de preguntas por cuestionario. 
\subsection{Zoho Survey}
Herramienta de creación de cuestionarios online con una versión gratuita con las siguientes características: 
\begin{itemize}
\item Número ilimitado de cuestionarios. 
\item 15 preguntas posibles por cuestionario. 
\item 150 respuestas posibles por cuestionario. 
\end{itemize}
Sin embargo, hemos descartado dicha herramienta por la limitación del número de preguntas por cuestionario. 
\subsection{Eval \& Go}
Herramienta de creación de cuestionarios online con posibilidad de creación de una cuenta gratuita con las siguientes características: 
\begin{itemize}
\item Número de encuestas ilimitadas por cuenta. 
\item Posibilidad de exportación de los  datos  recogidos en formato Excel o CSV. 
\end{itemize}
En un principio consideramos utilizar dicha herramienta, sin embargo, tiene un máximo de 150 respuestas al mes, y al necesitar el mayor número de respuestas de alumnos, tendríamos esa limitación,por lo que no podríamos recoger el máximo número de respuestas. \nocite{carl:encuestas}
\subsection{TypeForm}
Es una herramienta de creación de cuestionarios online anónimos, con una versión gratuita con las siguientes características: 
\begin{itemize}
\item Número ilimitado de preguntas, respuestas y cuestionarios por cuenta. 
\item Posibilidad de utilizar alguna plantilla predefinida o desarrollar una propia. 
\item Formularios responsive (adaptados a los diferentes dispositivos con los que se interaccione) 
\item Exportación de los datos en formato Excel en Drive o Google+, de forma que se puedan ver los resultados en tiempo real. 
\end{itemize}
El cuestionario realizado se encuentra en la siguiente dirección \url{https://clarapalacios.typeform.com/to/RQRRfY}. Es un cuestionario sencillo,  en donde se da una explicación breve del funcionamiento del mismo, se solicitan las ponderaciones por cada asignatura (obligatorias para los 3 primeros cursos académicos y optativas para el 4º año). No se permite que mismo usuario responda más de una vez al cuestionario, ya que, a pesar de ser anónimo, se almacena un identificador para evitar que un mismo alumno pueda rellenar varias veces el cuestionario.  	


\section{Integración de funcionalidades del cuestionario}
Los datos se guardarán en un documento Excel-sincronizado en Drive-  llamado  \textbf{DatosTFG\_SistemasRecomendacion}, almacenado en la siguiente dirección: \url{https://docs.google.com/spreadsheets/d/1Dtu6HPKu_d0zToz2b16dhfz4k7OluqAASYbveHUfSls/edit?usp=sharing} \\En un principio, teníamos pensado utilizar la librería de Python  OpenPyxl  de forma que se nos permitirá leer y almacenar en una matriz los datos recogidos del Excel. Sin embargo, nos encontramos problemas con la lectura, por lo que hemos decidido utilizar el API de GoogleDrive, descargando un fichero .json para la obtener la clave privada y almacenarla en el ordenador, en el directorio en donde se encuentre el código en Python de la lectura de datos de cuestionario anónimo. \\Además, serán necesarias dos librerías, \textbf{oauth2client} y \textbf{gspread}, pudiéndose instalar con un único código en cmd (Símbolo de Sistema de Windows)  mediante el comando \textbf{``pip install gspread oauth2client''}. \nocite{twilio:api}con la versión de pip 8.1.2, a pesar de no ser la más actual existente. 

\subsection{Funcionamiento del Google Drive API }
La activación e integración del fichero situado en Drive para poder utilizarlo desde Jupyter se ha realizado de la siguiente manera: 
\begin{itemize}
\item Iniciar sesión en Google Drive
\item Activar en la consola de APIs de Google la API. 
\item Crear (o en caso de tener uno creado, abrir) un nuevo proyecto. Hay un máximo de 11 proyectos. 
\item Activar las credenciales de Google Drive. 
\item Crear credenciales para Servidor Web para acceder a la aplicación. 
\item Dar un nombre al proyecto y asignarle un rol (Project-Editor)
\item Descargar el fichero .JSON correspondiente. 
\item Copiar el fichero (renombrado a client\_secret.json) en el directorio en donde obtengamos los datos en python del Excel. 
\item Copiar el client\_email situado en el fichero, y compartirlo con el fichero en Drive.
\item Instalar ambas librerías, e importarlas en el Notebook  de Python. 
\end{itemize}

\section{Herramienta para el desarrollo}
\subsection{Swagger}
Swagger es un framework de código abierto para desarrollar y visualizar las APIs REST, enganchando el código del sistema con el servidor de forma qeu se pueda generar automáticamente las respuestas basándose en el código de la aplicación. \nocite{wiki:swagger}
Este framework soporta un gran número de lenguajes de programación, entre los que podemos destacar PHP, JavasCript, Java, Python... Por lo que nos pareció correcto para las pruebas de buenas prácticas de la aplicación, documentar los métodos de los usuarios así como realizar pruebas de interfaz.\nocite{wiki:swagger} 

\subsection{PyQt5}
PyQt es un enlace de desarrollo multiplatforma de Python  \cite{python:pyqt} para la implementación y el desarrollo de interfaces gráficas. Utilizaremos la versión 5.6 de PyQt. Al ser un sofware libre, y con posibilidad de acceso a las bases de datos, consideramos que es de las mejores opciones para nuestro proyecto. \nocite{wiki:pyqt}

\subsection{PyDev}
PyDev es un IDE de Eclipse para el desarrollo, depuración y análisis de código en Python, de forma que, se permite no sólo  la programación en Python, sino en diferentes lenguajes. Por otra parte, permite la inclusión de test Unitarios, lo que es una gran ventaja a la hora de testear el código. 

\subsection{PythonAnywhere}
PythonAnyWhere es un IDE y servicio de alojamiento en Cloud que utiliza Python como lenguaje de programación. Permite el alojamiento tanto de código como de bases de datos y su acceso a través de una interfaz de comandos. Este IDE es compatible con la versión de Python utilizada, además de utilizar alojamiento web Flask (empleado en este proyecto)\nocite{wiki:pythonanywhere}


\subsection{Flask}
Flask es un frame que no necesita bibliotecas ni herramientas particulares para su utilización, escrito en Python para el alojamiento de BD en web basado en WSGI. 
Entre sus ventajas, podemos indicar: 
\begin{itemize}
\item Contiene un soporte de test unitarios. 
\item Contiene servidor tanto de desarrollo como de depuración. 
\item Licencia de software libre permisivo (BSD)que permite su utilización en un tiempo determinado. 
\item Es multiplataforma. 
\end{itemize}

\subsection{SQLAlchemy}
Herramienta de código abierto para el mapeo relacional de las bases de datos. Escrito en Python, perite el acceso a los datos de forma eficiente y con buen rendimiento gracias a la utilización el patrón del mapeador de datos en lugar del patrón de registro activo \cite{wiki:sqlalchemy}

\capitulo{5}{Aspectos relevantes del desarrollo del proyecto}
En este apéndice, incluiremos los puntos más relevantes del proyecto y las decisiones tomadas. 
\section{En Cloud}
La base de datos se encuentra localizada en Cloud en PythonAnyywhere, de forma gratuita. A pesar de haber subido únicamente la base de datos, se permitía incluir el código, y devolver un json con el resultado de la ejecución de los sistemas de recomendación para poder ser utilizado desde cualquier lenguaje. Sin embargo, dada la carga de peticiones realizada y la limitación por ser una cuenta gratuita, hemos considerado la opción de ejecutar el código en local, de forma que no se pueda sobrecargar el servidor. 

\section{Novedoso}
Este trabajo es novedoso, ya que es la primera vez que se realiza un sistema de recomendación para un Trabajo Final de Grado, por lo que consideramos esta idea como interesante, tanto como proyecto como su aplicación para los alumnos.  En un futuro se podría incluir un mayor número de sistemas de recomendación, así como cálculos de diferencias entre ellos. 

\section{LOPD}
Se cumplen de forma estricta las normativas de la LOPD, de forma que al servidor no se suben datos de carácter personal de los usuarios. Por otra parte, las contraseñas de los usuarios se encriptan, para prevenir el robo de las mismas por acceso no autorizado al servidor. 
\\En la misma línea, se ha realizado una aplicación de escritorio, ya que a los usuarios no suele gustar publicar sus datos en páginas web. Para prevenir el almacenamiento de los datos en el equipo, por si cualquier usuario decidiese eliminarlos y/o modificarlos,los datos del usuario se suben a servidor para no acceder a los ficheros directamente. 


\section{Datos de entrenamiento}
Los datos de entrenamiento para los sistemas de recomendación se han obtenido de un cuestionario realizado a los antiguos alumnos del grado de Ingeniería Informática, siendo distribuidos por los propios alumnos, profesores y el equipo de la Asociación de Ingenieros Informáticos quienes lo distribuyeron entre los alumnos de promociones anteriores. En este aspecto, me gustaría agradecer a la directiva de la Asociación de Ingenieros Informáticos, y en especial a  Javier López Martínez  por haber hecho posible dicha publicación. 





\capitulo{6}{Trabajos relacionados}
Los sistemas de recomendación son una herramienta muy utilizada actualmente, marcando un punto de inflexión tanto  en el comercio electrónico   como  en las diferentes webs y apps, tales como Youtube, Spotify... 
Sin embargo, en relación a diferentes proyectos de Trabajo Final de Grado, éste es el primer proyecto orientado a dicho campo. 

Entre las diferentes plataformas, podemos especificar:

\section{Amazon}
Amazon utiliza el sistema de recomendación "ítem to ítem colaborative filtering" inventado por la propia empresa en 1998 ante el problema de los filtros colaborativos basados en usuarios:
\begin{itemize}
\item Fallo en la recomendación al disponer de una gran cantidad de elementos y pocas votaciones
\item Gran coste en el cálculo de las similitudes de usuarios. 
\item Cambio de perfil de usuarios de forma rápida.\nocite{wiki:filter}
\end{itemize}
Por otro lado, utiliza un sofware dirigido por un equipo de trabajadores encargado de enviar ofertas específicas a los diferentes usuarios según las últimas compras realizadas.\nocite{manu:amazon}
\section{Spotify}
Spotify  utiliza 3 sistemas de recomendación diferentes para mejorar las recomendaciones de sus canciones en Discovery Weekly: 
\begin{itemize}
\item Filtro colaborativo basado en usuarios. 
\item Modelo basado en el audio en bruto: Sistema de recomendación que utiliza una red neuronal (comvolutional neural network) para obtener las características musicales de las canciones escogidas por el usuario. 
\item NLP Sistema de procesamiento de lenguaje natural para encontrar nuevas tendencias musicales con las que poder trabajar. \cite{isaac:spotify}
\end{itemize}



\section{Youtube}
El sistema de recomendación de Youtube es bastante simple, ya que utiliza un filtro colaborativo basado en memoria (basado en usuarios). Sin embargo, lo filtra mediante calificaciones de forma que se mejora la ponderación del vídeo en los siguientes aspectos: 
\begin{itemize}
\item Regionalización: Filtrando el FC en el área geográfica en donde se encuentre el usuario o  en  el lenguaje de las búsquedas previas realizadas. \nocite{md:youtube}
\item Temporización: El sistema de recomendación de Youtube favorece los vídeos con una visualización de mayor duración, sin importar el porcentaje del vídeo reproducido. \nocite{anali:youtube}
\item Número de vídeos vistos por los usuarios a partir de uno determinado en una misma sesión. \nocite{hoots:youtube}
\item Determina si un vídeo gusta o no a los clientes utilizando inteligencia artificial para observar los comentarios inscritos en el vídeo. 
\end{itemize}

\include{./tex/7_Asignaturas_Ingenieria_Informatica}
\capitulo{8}{Conclusiones y Líneas de trabajo futuras}
En este capítulo indicaremos las conclusiones obtenidas del proyecto, de su realización y de las posibilidades, así como las líneas por las que se podría seguir su desarrollo. 
\section{Conclusiones}
Este proyecto ha sido novedoso, tanto por ser un  sistema de recomendación, como su implementación  y desarrollo. Ha sido un reto el tener el servidor en Cloud y poder acceder a los datos desde una aplicación de escritorio. En este proyecto también sería posible utilizar un servidor local, en caso de fallo de conexión a Internet. 
A pesar de la dificultad de adaptación del código para que almacene los datos recogidos a través de la interfaz y los almacene en la Base de Datos, se  ha conseguido dicha adaptación de la forma más limpia posible. 
Por otro lado, se ha deseado implementar una mayor variedad de sistemas de recomendación con diferentes algoritmos, sin embargo, a falta de tiempo, únicamente se han desarrollado dos. 
También se ha deseado tener el código de los sistemas de recomendación alojados en Cloud, pero, ante la falta de recursos de forma gratuita disponibles en el servidor, no hemos querido sobrecargarlo con peticiones, por lo que se ha dejado el código en local. 

\section{Líneas futuras de desarrollo}
En un futuro, se podría continuar con el desarrollo de los siguientes aspectos: 
\begin{itemize}
\item Almacenamiento del código de los sistemas de recomendación en Cloud para permitir implementar la interfaz con diferentes lenguajes de programación. 
\item Desarrollo de diferentes algoritmos para los sistemas de recomendación, para ofrecer mayores puntos de vista al usuario. 
\item Desarrollo del rol administrador con una interfaz diferente para el acceso a la BD, y la modificación, agregación y eliminación de asignaturas. 
\item Desarrollo de una interfaz gráfica en Web. 
\item Desarrollo de  nuevas pestañas en la interfaz, tales como diferencias entre sistemas de recomendación o personalización de perfiles de usuarios o hacer un top de ejecuciones en modelos no deterministas. 
\item Mostrar  las menciones que pertenecen a las  recomendaciones ofrecidas al usuario. 
\item Ampliar a todos los cursos los sistemas de recomendación, no sólo a las asignaturas del cuarto curso. 
\item Personalizar el número de asignaturas de la recomendación, ya que en el cuarto curso no son obligatorias las diez asignaturas ya que un usuario puede no necesitar todas las asignaturas. 
\item Desarrollar una pestaña para priorizar las menciones frente a las recomendaciones, si es ése el interés del alumno. 
\item Aplicar el sistema de recomendación a diferentes grados universitarios, de forma que el sistema no sea excluyente para el grado de Ingeniería Informática. 
\end{itemize}


\bibliographystyle{plain}
\bibliography{bibliografia}

\end{document}